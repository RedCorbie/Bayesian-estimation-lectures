$$ X \in \setR, p(x|\theta), \hat{\theta} \in \Theta, \pi(\hat{\theta}) $$
Какое может быть семейство распределений на $x=\braces{1,2}$? Все элементы эквиваленты точке в области $\braces{\hat{p}=(p_1,p_2), p_1, p_2 \geq 0, p_1+p_2=1}, \ p_1=\prob{x=1}, p_2=\prob{x=2}$. Надо ввести априорное распределение на $p_1$ и $p_2$. Пример: бета-распределение
\begin{definition}
	С.в. распределена с $\sim \beta$-распределением с пар-рами $\alpha_1, \alpha_2 > 0$, если $$ \prod (p_1) = \frac{\Gamma(\alpha_1 + \alpha_2)}{\Gamma(\alpha_1)\Gamma(\alpha_2)} p_1^{\alpha_1-1} (1-p_1)^{\alpha_2-1}, \ 0 \leq p_1 \leq 1$$
	Если $\alpha_1 \to 0 \Rightarrow \prob{p_2=1}=1$
\end{definition}

Пусть есть нез. с.в. $z_1, z_2$ с распр. $\Gamma$ с пар-рами $\alpha_1, \alpha_2$, т.е. $p(z_i) = \frac{1}{\Gamma(\alpha_1)} \exp{-z_i}z_i^{\alpha_i-1}, \ z_i > 0$
$$ \frac{z_1}{z_1+z_2} \sim \beta(\alpha_1, \alpha_2) $$
$$ x_1, \ldots, x_n \in \braces{1,2}, \hat{p} = (p_1, p_2)$$
$\Rightarrow \beta(\alpha_1, \alpha_2)$ - сопряженное в том смысле, что если у нас есть $p(\hat{p}|\hat{x}) = \beta(\alpha_1',\alpha_2')$, где 
$$ \alpha_1' = \alpha_1 + \sum \squarebraces{x_i == 1}, \ \alpha_2' = \alpha_2 + \sum \squarebraces{x_i == 2}$$
Рассмотрим более общий случай: $x_i \in \setX = \braces{1, \ldots, k}$. Все распределения = множеству векторов $\braces{\hat{p} = (p_1, \ldots, p_k), p_i \geq 0, \sum p_i = 1}, \ \prob{x=i}=p_i$
\begin{definition} Распределение Дирихле \\
	$\bar{\alpha} = \braces{\alpha_1, \ldots, \alpha_k}, \alpha_i \geq 0$
	$$ \prob (p_1, \ldots, p_{k-1}) = \frac{\prod(\sum \alpha_i)}{\prod_{i=1}^k \Gamma(\alpha_i)}\prod_{i=1}^k p_i^{\alpha_i-1} (1-\sum_{i=1}^{k-1} p_i)^{\alpha_k-1}$$
\end{definition}

Свойства:
\begin{enumerate}
	\item $z_1, \ldots, z_k, \ z_i \sim \Gamma(\alpha_i)$ - независимые. Тогда вектор $\big( \frac{z_1}{\sum z_i}, \ldots \frac{z_k}{\sum z_i} \big) \sim D(\hat{\alpha})$
	\item $A_1, \ldots, A_n \subset \setX, A_i пересеч. A_j = непуст., UA_i = \setX=\braces{1, \ldots, k} \Rightarrow \big( P(A_1), \ldots, P(A_m) \big) \sim D(\alpha_1', \ldots, \alpha_m'), \ \alpha_i' = \sum_{j \in A_j} \alpha_j$ \\
	\item Удобно говорить о $\alpha(A)$ как о мере: $ \alpha(A) = \sum_{i \in A_i} \alpha_i$. Тогда можно переписать: $$ \sim D(\alpha(A_1), \ldots, \alpha(A_m)) $$
	\item $\prob(X_{n+1}=k|X_1, \ldots, X_n) = \frac{\alpha_k + \sum_{i=1}^n [x=k]}{\alpha(\setX)+n}, \ p(p) = D(\hat{\alpha})$
	$$ p(\hat{p}|\hat{x}) = D(\hat{\alpha}'), \alpha_i' = \alpha_i + \sum_{j=1}^n [x_j=i] $$
\end{enumerate}

Пример с шарами: 
$$ \prob{x_1=j} = \frac{\alpha(j)}{\sum_{i=1}^n \alpha(i)} $$
$$ \prob{x_2=j|x_1} = \frac{\alpha(j) + [x_1=j]}{\alpha(\setX) + 1} $$
$$ \prob{x_1=X_1, \ldots, x_n=X_n} = \frac{\alpha(X_1)}{\alpha(\setX)} \Big( \frac{\alpha(X_2) + [x_1=x_2]}{\alpha(\setX)+1}\Big) \ldots 
\frac{\alpha(X_n) + \sum_{i=1}^{n-1}[x_i=x_n]}{\alpha(\setX)+n-1} $$

Можно расширить на случай $\setX=\setR^d, \ M(\setX)$. Вводим в качестве априорного распределения распределение Дирихле. Если задана некоторая мера $H$ на множестве $\setR^d$ и некоторое $\alpha>0$, то $\exists !$ вероятностная мера  $D_{\alpha, H}$ на $M(\setR^d)$ т.ч. при любом разбиении $\setR^d$ на борелевские мн-ва $B_1, \ldots, B_m$:
$$ (P(B_1), \ldots, P(B_m)) \sim D(\alpha H(B_1), \ldots, \alpha H(B_m)) $$, если $\alpha>0, H$ - конечная мера. Тогда это случайный процесс Дирихле

\subsection*{Сравнение гауссовского процесса и процесса Дирихле}
Для гауссовского процесса берем $x \in \setR^d$. Для процесса Дирихле $A \in B(\setR^d)$.
Реализация процесса:
1)$ f: \setR^d \to \setR $, 
2)$ G$ - вероятностная мера
Параметры процесса:
1) среднее $\mu(\hat{x})$, ковариационную функцию $k(\hat{x}, \hat{x}')$. 
2) Задать вер. меру $H$ на $\setR^d$ и некоторое $\alpha>0$
Одномерная проекция:
1) $f(\hat{x}) \sim N(\mu(\hat{x}), k(\hat{x}, \hat{x}'))$ 
2) $G(A) \sim \beta(\alpha H(A), \alpha (1-H(A)))$
Многомерный срез:
1) $f(\hat{x}_1), \ldots, f(\hat{x}_n) \sim N((\mu(\hat{x_1}), \ldots \mu(\hat{x_n})), K)$ 
2)$A_1, \ldots A_n$. $G(A_1) ,\ldots, G(A_n) \sim D(\alpha H(A_1), \ldots, \alpha H(A_n))$


Важно!
\begin{enumerate}
	\item С вероятностью 1 реализация процесса Дирихле – это дискретная вероятностная мера: $p(x) = \sum_{i=1}^\infty p_i \delta_{x_i}(x), \ p_i >0, \sum p_i=1$
	\item Рассмотрим модель: $G \sim D(\alpha, H)$. Генерируем из вер. меры набор $x_1, \ldots x_n | G \sim G$. Можем явно выписать апостериорное распределение: $G | x_1, \ldots x_n \sim D(\frac{\alpha}{\alpha+n}H + \frac{1}{\alpha+n}\sumi \delta_{x_i}, \alpha+n)$
	$$ x_n+1 | x_1 \ldots, x_n \sim \frac{\alpha}{\alpha+n}H + \frac{1}{\alpha+n}\sumi \delta_{x_i}$$
\end{enumerate}


\begin{definition}[Процесс «китайский ресторан»]
	$x_1, \ldots x_n$ - уже есть. $x_{n+1}|x_1, \ldots, x_n \sim H$ с вер-тью $\frac{\alpha}{\alpha+n}$, $\delta_c$ с вероятностью $\frac{\sum[x_j=c]}{\alpha+n}$
\end{definition}
Получаем условно кластеры. Можем посчитать матожидание их числа. Пусть на шаге $n$ число кластеров - $m$.
$$ \setE_{\alpha, n}m = \alpha(\psi(\alpha+n) + \psi(\alpha)) = \alpha \log (1 + \frac{n}{\alpha}), n \to \infty $$

\subsection*{Смесь распределений с априорным распределением, заданным процессом Дирихле}
Смесь распределений: $p(\hat{x}|\hat{\theta}), \hat{\theta} \in \setR^p$. $\hat{\theta}$ - априорн. распределение G, G - реализация процесса Дирихле.
$$ G \sim D(H, \alpha), \theta_i \sim G, \ x_i \sim p(x_i|\theta_i), \ i=1,\ldots,n $$. 
Модельный пример: 
\begin{example}
	Пусть $p(x|\theta) = N(\hat{\mu}, \sigma_X I), \hat{\mu} = \theta, \sigma_x$ - известно. \\
	$p_H(\theta) = N(0, \sigma_{\mu}), \sigma_{\mu}$ - известно. Тогда
	$$ p(\mu|x_1, \ldots, x_n) = N(\frac{\sigma_n}{\sigma_X} \mu, X_i, \sigma_n), \frac{1}{\sigma_n} = \frac{1}{\sigma_{\mu}} + \frac{1}{\sigma_X} $$
	$$ p(x) = \int p(x|\mu)p_H(\mu)d\mu = N(0, \sigma_{\mu} + \sigma_X) $$
\end{example}
Проведем сэмплирование Гиббса:
$\braces{\hat{\theta_i}}, i=1, \ldots, n$. $$ p(\theta_i | \hat{\theta}_{\\i}, \hat{x}), \hat{\theta}_{\\i} = braces{\hat{\theta_j}}_{j \neq i} $$

$$ p(\theta_i | \hat{\theta}_{\\i}), \hat{x}) = \frac{p(\theta_i | \hat{\theta}_{\\i})) p(x_i|\theta_i)}{\int p(\theta_i | \hat{\theta}_{\\i})) p(x_i|\theta_i) d\theta_i} $$
$$ p(\theta_i | \hat{\theta}_{\\i})) = \frac{\alpha}{\alpha+n} p_h(\theta_i) + \frac{1}{\alpha+n-1} \sum_{i\neq j} \delta_{\theta_j}(\theta_i) $$
$$ p(\theta_i | \hat{\theta}_{\\i}), \hat{x}) = \frac{1}{z} (\alpha (\int p(x_i||\theta)p_H(\theta) d\theta) p(\hat{\theta}_i|x_i)) +
\sum_{j \neq i} \delta_{\theta_j}(\theta)p(x_i|\theta_j))$$
$$ p(\mu_i|\hat{\mu_{x_i}}, \bar{x}) = \frac{1}{z} \alpha N(x_|0, (z_{\mu} + z_x)) N(\mu_i|\frac{\sigma_1}{\sigma_x}x_i, \sigma_1) + 
\frac{1}{z}  \sum \delta_{\mu_j}(\mu)N(x_i|\mu_j, \sigma_x)$$

Генерировать распределение Дирихле можно еще одним способом: «ломание палки»
$$ p(\bar{p}) = p(p_1)p(p_2|p_1)p(p_3|p_2, p_1) $$
$$ p_1 = v_1 \sim \beta(\alpha_1, \sum_{i=2} \alpha_i), v_2 \sim \beta(\alpha_2, \sum_{i=3} \alpha_i) ... $$
$$ p_k = 1 - \sum_{i=1}^k p_i, p_i = v_i \prod_{j=1}^{i-1}(1-v_j) $$
