\begin{definition}
	Регулярность $p(\theta|x)\Pi(\theta), \ \theta \in \Theta$:
	\begin{enumerate}
		\item $\braces{x: p(x|\theta)>0}$ одинаков $\forall \theta \in \Theta$
		\item $L(\theta, x) = \log p(\theta|x)$ трижды дифференцируема по $\theta$ в окрестности $(\theta_0-\delta, \theta_0+\delta)$:
		\begin{itemize}
			\item $\set{E}_{\theta_0} L'(\theta_0), \ \set{E}_{\theta_0} L''(\theta_0)$ - конечны
			\item $\sup_{\theta \in (\theta_0-\delta, \theta_0+\delta)} |L'''(\theta, x)| < M(x), \ \set{E}_{\theta_0}M < \infty$ 
		\end{itemize}
		\item $\set{E}_{\theta_0} L(\theta_0) = 0$, \\ $\set{E}_{\theta_0} L''(\theta_0) = - \set{E}_{\theta_0} (L'(\theta_n))^2$
		\item $I(\theta_0) = \set{E}_{\theta_0} (L'(\theta_0))^2 > 0$
	\end{enumerate}
\end{definition}

\begin{theorem}
	Пусть $p(\theta|x)$ удовлетворяет 1)-4) и $\valuation$ - состоятельная оценка максимального правдоподобия \\
	$\Rightarrow$ $$\frac{1}{\sqrt{n}} (\valuation - \theta_0) \overset{D}{\to} N(0, \frac{1}{I(\theta_0)})$$

	\begin{proof}
		$0 = L'_n(\valuation), \ \valuation$ - ОМП, $L_n(\theta) = \sum_{i=1}^nL(\theta, x_i)$ \\
		$$0 = L'_n(\theta_0) + (\valuation-\theta_0)L''_n(\theta_0) + \frac{(\valuation - \theta_0)^2}{2} L'''_n(\theta'), \ \theta_0<\theta'<\valuation$$
		$$ \sqrt{n}(\valuation-\theta) = \frac{\frac{1}{\sqrt{n}}L'_n(\theta_0)}{-\frac{1}{n}L''_n(\theta_0) - \frac{1}{2}(\valuation-\theta_0)\frac{1}{n}L''(\theta')} $$
		Из ЦПТ:
		$$ \frac{1}{\sqrt{n}}L'_n(\theta_0) \overset{D}{\to} N(0, I(\theta_0)) $$
		Из УЗБЧ:
		$$ -\frac{1}{n} L'_n(\theta_0) - \frac{1}{2}(\valuation-\theta_0)\frac{1}{n}L''_n(\theta') \to i(\theta_0) $$
	\end{proof}
\end{theorem}

\begin{definition}
	Для байесовского случая дополнительные условия:
	\begin{enumerate}
		\item[5.] $\forall \delta > 0 \exists \varepsilon: \set{P}_{\theta_0}\braces{\sup_{|\theta-\theta_0|>\delta} \frac{1}{n}(L_n(\theta) - L_n(\theta_0)) \leq \varepsilon} \to 1$
		\item[6.] $\pi(\theta)$ - априорное распределение - непрерывно и в $\theta_0>0$
	\end{enumerate}
\end{definition}

\begin{theorem}
	Пусть 1)-6) выполнены. Тогда: 
	$$ \int_{\setR} |\pi^*(S|x) - \frac{\sqrt{I(\theta_0)}}{\sqrt{2\pi}}\exp\braces{-\frac{S^2}{2}I(\theta_0)}|ds \overset{\set{P}_{\theta_0}}{\to} 0 $$
\end{theorem}

\begin{theorem}[Бернштейна-фон Мизеса]
	$$ \int \pi(\theta)|\theta|  d\theta < \infty $$
	$$ \theta_n^* = \int_{\setR} \theta p(\theta|x) d\theta $$ - байесовская оценка \\
	Тогда:
	\begin{enumerate}
		\item $ \sqrt{n} (\valuation - \theta_n^*) \overset{\set{P}_{\theta_0}}{\to} 0 $
		\item $ \sqrt{n}(\theta_n^* - \theta_0) \overset{D}{\to} N(0, \frac{1}{I(\theta_0)}$
	\end{enumerate}
\end{theorem}

\subsection*{Априорное распределение Джеффри}
$$ \pi_j(\theta) \propto \sqrt{I(\theta)} $$
$$ I(\theta) = -\set{E}_{\theta_0} \braces{\frac{\partial^2 \log p(x|\theta)}{\partial \theta^2}} $$

\subsection*{Опорное априорное распределение}
$$ \pi^*(\theta) = \arg \max_{\substack{\pi(\theta)}} MI(\Theta, T) $$
T - достаточная статистика $X_1, \ldots, X_n$
$$ MI(\Theta, T) = \int p(t) \int p(\theta, t) \log \frac{p(\theta | t)}{\pi(\theta)} d\theta dt = \int p(t) KL(p(\theta|t)|| \pi(\theta)) dt $$

\begin{example}
	$$ X \sim N(\theta, I), \ x \in \setR^p, \  \ ||\theta||^2 - ? $$
	\solution{
		$$\prob{||theta||^2|X} \sim \Xi^2(p) + ||X||^2$$
		$$ \set{E}_{\theta}\braces{||\theta||^2|X} = ||X||^2 + p $$
		Для опорного распределения Джеффри $$ ||\theta||^2_{prob} $$ - оценка с минимальной дисперсией
		$$ ||\theta||^2_{prob} = ||X||^2-p $$
	}
\end{example}

$T^k$ - набор статистик. $$\pi_k(\theta) = \arg \max_{\substack{\pi(\theta)}} MI(\Theta, T^k)$$
$$ MI(\Theta, T^k) = \int p(t^k) KL(p(\theta|t^k) || \pi(\theta)) dt^k = \int \pi(\theta) \log\frac{f_k(\theta)}{\pi(\theta)}d\theta $$
где $$ f_k(\theta) = \exp \braces{\int p(t^k|\theta) \log p(\theta|t^k) dt^k} $$
Для бесконечного числа экспериментов:
$$ \pi^*(\theta) = \lim_{\substack{k \to \infty}} \pi_k(\theta) $$

\begin{example}
$$ \pi(\theta) \to \braces{p_i}, \ \Theta = \braces{\theta_1, \ldots, \theta_k}, q=f_k(\theta) $$
$$ \arg \max_{\substack{p, \lambda}} \sum p_i \log \frac{q_i}{p_i} + \lambda(\sum p_i-1) - ? $$
\solution{
	Через метод множителей Лагранжа:
	$$ (\log\frac{q_i}{p_i} + p_i \frac{q_i}{p_i}(-\frac{q_i}{p_i^2})) + \lambda = 0 $$
	$$ \log\frac{q_i}{p_i} = \lambda - 1 $$
	$$ \frac{q_i}{p_i} = e^{\lambda-1} \ \Rightarrow p_i=q_ie^{\lambda-1} $$
	$\pi_i(\theta) \sim f_k(\theta) $ - в непрерывном случае
}
\end{example}

\begin{example}
Пусть $\hat{\theta}_k$ удовлетворяет теореме Бернштейна-фон Мизеса:
$$ || p(\theta|\hat{\theta}_k) - N(\hat{\theta}_k, \frac{1}{I_k(\breve{\theta}_n)}) || \to 0 $$
$$ I^{-1}_k(\theta_0) = \frac{1}{k}I^{-1}(\theta_0) $$
$$ p(\theta|\hat{\theta}_k) \propto \sqrt{k} \sqrt{I(\hat{\theta}_k)} \exp \braces{-\frac{I(\hat{\theta}_2)}{2k}(\theta - \hat{\theta}_k)^2} $$
Если $\theta_0$ - истинное значение:
$$  p(\theta|\hat{\theta}_k) \approx \sqrt{k} \sqrt{I(\hat{\theta}_k)} \exp \braces{-\frac{I(\hat{\theta}_2)}{2k}(\theta - \theta_0)^2} = \sqrt{kI(\theta_0)}$$
$\Rightarrow$
$$ f_k(\theta) = \exp \braces{\int p(t^k|\theta) \log \sqrt{I(\theta)} dt^k} = \sqrt{I(\theta_0)} \exp \braces{\int p(t^k|\theta) dt^k} = \sqrt{I(\theta_0)} $$
$\Rightarrow$ в одномерном случае опорное априорное распределение и априорное распределение Джеффри совпадают (но в многомерном - нет)
	
\end{example}