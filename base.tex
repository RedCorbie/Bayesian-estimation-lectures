\documentclass[a4paper]{article}

\usepackage[utf8]{inputenc}
\usepackage[english, russian]{babel}
\usepackage[fleqn]{amsmath}
\usepackage{amsfonts, amssymb, amsthm, mathtools}
\usepackage{amsmath}

\usepackage{mathtools}
\usepackage{fullpage}
\usepackage[colorinlistoftodos]{todonotes}

\usepackage{enumitem}
\usepackage{tikz} 

\title{Байесовские методы статистического оценивания}
\author{ИППИ}


\theoremstyle{plain}
\newtheorem*{theorem-star}{Theorem}
\newtheorem{theorem}{Theorem}
\newtheorem*{lem-star}{Lemma}
\newtheorem{lem}{Lemma}
\newtheorem*{proposition-star}{Proposition}
\newtheorem{proposition}{Proposition}
\newtheorem{statement}{Statement}
\newtheorem*{statement-star}{Statement}
\newtheorem{corollary}{Corollary}
\newtheorem*{corollary-star}{Corollary}

\theoremstyle{remark}
\newtheorem*{remark}{Remark}

\theoremstyle{definition}
\newtheorem*{definition-star}{Definition}
\newtheorem{definition}{Definition}
\newtheorem{example}{Example}
\newtheorem*{example-star}{Example}
\newtheorem{problem}{Problem}
\newtheorem{problem-star}{Problem}

\newcommand{\solution}[1]{\textbf{Solution: } \\ #1 $\bullet$}

\renewenvironment{proof}{{\bfseries Proof}}{$\bullet$}

\newcommand{\walls}[1]{\left | #1 \right |} % |smth_vertically_large|
\newcommand{\braces}[1]{\left\{ #1 \right\}} % {smth_vertically_large}
\newcommand{\squarebraces}[1]{\left[ #1 \right]}

\newcommand{\myequat}[1]{\begin{equation} #1 \nonumber \end{equation}}
\newcommand{\pars}[1]{\left( #1 \right)} 
\newcommand{\class}[1]{\left[ #1 \right]} 
\newcommand{\dd}{\; \mathrm{d}}
\newcommand{\setR}{\mathbb{R}}
\newcommand{\setRn}{\mathbb{R}^n}
\newcommand{\setRinf}{\mathbb{R}^{\infty}}
\newcommand{\setC}{\mathbb{C}}
\newcommand{\setN}{\mathbb{N}}
\newcommand{\setZ}{\mathbb{Z}}
\newcommand{\setQ}{\mathbb{Q}}
\newcommand{\setM}{\mathcal{M}}
\newcommand{\setL}{\mathcal{L}}
\newcommand{\setA}{\mathcal{A}}
\newcommand{\setF}{\mathcal{F}}
\newcommand{\setX}{\mathcal{X}}
\newcommand{\setE}{\mathcal{E}}
\newcommand{\set}[1]{\mathbb{#1}}
\newcommand{\ex}[1]{\mathbb{E}\braces{#1}}
\newcommand{\prob}[1]{\mathbb{P}\braces{#1}}

\newcommand{\valuation}{\hat{\theta}_n}
\newcommand{\sumi}{\sum_{\substack{i=1}}^n}

%\newcommand*\circled[1]{\tikz[baseline=(char.base)]{\node[shape=circle,draw,inner sep=2pt] (char) {#1};}}
\newcommand*{\circled}[1]{%
\footnotesize\protect\tikz
\protect\node[shape=circle,draw,inner sep=2pt,line width=0.2mm](n1){#1};}



\newcommand{\condset}[2]{\braces{\, #1 \mid #2 \,}} % definition of set with condition

\newcommand{\expl}[1]{\walls{\text{#1}}} % explanation inside formula

\newcommand{\toup}[1]{\xrightarrow{#1}}
\newcommand{\toae}{\toup{\text{\,п.н.}}} % almost everywhere convergence designation
\newcommand{\todown}[1]{\xrightarrow[#1]{}}

\newcommand{\equp}[1]{\stackrel{#1}{=}}

\newcommand{\conj}[1]{\overline{#1}} % complex conjugation
\newcommand{\comp}[1]{\overline{#1}} % set complement

\newcommand{\notiff}{%
  \mathrel{{\ooalign{\hidewidth$\not\phantom{"}$\hidewidth\cr$\iff$}}}}

\newcommand{\notif}{%
  \mathrel{{\ooalign{\hidewidth$\not\phantom{"}$\hidewidth\cr$\Rightarrow$}}}}
\newcommand\indep{\protect\mathpalette{\protect\independenT}{\perp}}
\def\independenT#1#2{\mathrel{\rlap{$#1#2$}\mkern2mu{#1#2}}}
\newcommand{\argmax}{\arg\max}

\DeclareMathOperator{\cov}{cov}

\renewcommand{\emptyset}{\varnothing}
\renewcommand{\epsilon}{\varepsilon}
\renewcommand{\phi}{\varphi}
\renewcommand{\leq}{\leqslant}
\renewcommand{\geq}{\geqslant}
\renewcommand{\Im}{\mathop{\mathrm{Im}}\nolimits}
\renewcommand{\Re}{\mathop{\mathrm{Re}}\nolimits}

\DeclareMathOperator*{\argmin}{arg\,min}
\newcommand{\bX}{\textbf{X}}

\newcommand{\bigtitle}[1]{\title{\textbf{\underline{#1}}}}
\newcommand{\boldtitle}[1]{\title{\textbf{#1}}}

\newcommand*{\hm}[1]{#1\nobreak\discretionary{}%
{\hbox{$\mathsurround=0pt #1$}}{}} 


\begin{document}
\maketitle

\section*{02.09.15}
$$ \set{X} = \braces{X_1, \ldots, X_n}, x_i \in \set{X} \subset \setR^d $$
$$ \set{P}\braces{\set{X} | \theta}, \theta \in \Theta \subset \setR^p $$

Оценки:
\begin{itemize}
	\item точечные
	\item интервальные
\end{itemize}

Необходимые свойства:
\begin{itemize}
	\item Состоятельность: $\hat{\theta}_n \overset{P}{\to} \theta, n \to \infty$
	\item Несмещенность: $\set{E}\hat{\theta}_n = \theta$ (асимптотическое равенство в пределе)
\end{itemize}
Взаимосвязь:
\begin{itemize}
	\item $\hat{\theta}_n \overset{P}{\to} \theta \notif \set{E}\hat{\theta}_n = \theta$ {\Huge(}пример: $\theta=0, \hat{\theta}_n = \frac{1}{n}${\Huge)}
	\item $\hat{\theta}_n \overset{P}{\to} \theta \notif \set{E}\hat{\theta}_n \underset{\infty}{\to} \theta$ (пример: $ \hat{\theta}_n = 
	\begin{cases}
    0       & \text{if } p=1-\frac{1}{n}\\
    n^2  &\text{if } p=\frac{1}{n}\\
  \end{cases},
  \set{E}\hat{\theta}_n = n^2 \frac{1}{n} = n $)
\end{itemize}

Обратно:
\begin{itemize}
	\item $ \set{E}\hat{\theta}_n = \theta \notif \hat{\theta}_n \overset{P}{\to} \theta$ (пример:) $\hat{\theta}_n = \begin{cases}
    n^2       & \text{if } p=\frac{1}{2}\\
    -n^2  &\text{if } p=\frac{1}{2}\\
  \end{cases}$
	\item $\set{E}\hat{\theta}_n \underset{\infty}{\to} \theta \notif \hat{\theta}_n \overset{P}{\to} \theta $
\end{itemize}
\begin{example}
$$ x \sim Bern(\theta), \theta \in (0,1), n=1$$. \\ Оценить $\phi(\theta) = \frac{1}{\theta}$ - не существует несмещенной оценки. \\
\\
\solution{
	$\hat{\phi}(x)$ - оценка, зависящая от реализации. Для распределения Бернулли нужно задать $\big(\hat{\phi}(0), \hat{\phi}(1)\big)$ \\
	Несмещенность: $$ \set{E}\hat{\phi}(x) = \phi(\theta) = const $$
	С другой стороны: $$ \set{E}\hat{\phi}(x) = (1-\theta)\hat{\phi}(0) + \theta\hat{\phi}(1) - \infty $$
	Так как $\infty \neq const$, несмещенной оценки не существует
}

\end{example}

\begin{example}
$$ N(\mu, \sigma^2) $$
$$ \hat{\sigma}^2 = \frac{1}{n} \sum_{i=1}^n (x_i-\bar{x})^2 $$
Будет ли эта оценка смещенной, асимптотически смещенной? \\
\solution{
	$$\set{E}\hat{\sigma}^2 = \frac{1}{n}\set{E}\sum_{i=1}^n (x_i - \bar{x})^2 = \frac{1}{n}\sum_{i=1}^n \set{E} \braces{x_i^2 -2x_i\bar{x} + \bar{x}^2} = $$
	$$ =\frac{1}{n} \sum_{i=1}^n \sigma^2 - \frac{2\sigma^2}{n} + \frac{n\sigma^2}{n^2} = \sigma^2 - \frac{\sigma^2}{n} = \frac{n-1}{n} \sigma$$ - смещенная оценка. \\
	Несмещенную можно получить, домножив на $\frac{n}{n-1}$: $$ \frac{1}{n-1} \sum_{i=1}^n (x_i - \bar{x})^2 $$
}
\end{example}

\subsection*{Способы доказательства состоятельности статистических оценок}
\begin{enumerate}%[label=\circled{\arabic*}]
	\item Подсчитать функцию распределения
	\begin{example}
	$$ p(x) = \begin{cases}
	\frac{1}{\theta} & x\in(0, \theta) \\
	0 & иначе \\
	\end{cases} $$
	где $0 \leq \theta_1 \leq \theta \leq \theta_2$ \\
	\solution{
	Оценим $\theta$ с помощью n-й порядковой статистики: $\hat{\theta}_n = X_{(n)}$ \\
	Функция распределения: $$F(x) = \set{P}\braces{x_i \leq x} = \Big( \frac{x}{\theta}\Big)^n$$ \\
	Оценить вероятность: $$ \set{P}\braces{|\hat{\theta}_n - \theta| > \epsilon} = \set{P}\braces{X_{(n)} \leq \theta - \epsilon} = \Big( 1 - \frac{\epsilon}{\theta}\Big)^n \to 0 (n \to \infty) $$
	Наблюдение $>\theta$ невозможно, значит есть сходимость по вероятности
	}
	\end{example}
	\item Закон больших чисел
	\begin{itemize}
		\item $\frac{1}{n}\sum_{i=1}^n x_i \overset{P}{\to} \set{E}x_1$
		\item $\begin{cases}
			\hat{\theta}_n \overset{P}{\to} \theta \\
			\phi(t) \ непр. \ t=0 \\
		\end{cases}$ $\Rightarrow \phi(\hat{\theta}_n) \overset{P}{\to} \phi(\theta) $
	\end{itemize}

	\item Если смещение и дисперсия стремятся к 0, то оценка состоятельная \\
	\begin{proof} \\
	$$ \begin{cases}
			\set{E}(\hat{\theta}_n - \theta) \to 0 \\
			\set{E}(\hat{\theta}_n - \set{E}\hat{\theta}_n) \to 0 \\
		\end{cases} \Rightarrow \set{P}\braces{|\hat{\theta}_n - \theta| < \epsilon} \to 1 (n \to \infty) $$ 
		Докажем через неравенство Чебышева:
		$$ \set{P}\braces{|\hat{\theta}_n - \theta| > \epsilon} \leq \frac{\set{E}(\hat{\theta}_n - \theta)^2}{\epsilon^2}$$
		$$  \set{P} \braces{|\hat{\theta}_n - \theta| > \epsilon} \to 0$$
		$$ \set{E}(\hat{\theta}_n - \theta)^2 = \set{E}(\hat{\theta}n - \set{E}\hat{\theta}_n + \set{E}\hat{\theta}_n - \theta)^2 = \set{D}\hat{\theta}_n - 2 \set{E}(\hat{\theta}n - \set{E}\hat{\theta}_n)(\set{E}\hat{\theta}_n - \theta) + (\set{E}\hat{\theta}_n - \theta)^2 \to 0 $$
	\end{proof}
\end{enumerate}

$\Leftrightarrow$

\subsection*{Эффективные оценки} 
Функция риска минимальна: $ D\hat{\theta}_n \to \min$ \\
Оценка снизу: неравенство Рао-Крамера: $$ D\valuation \geq \frac{1}{nI_n(\theta)}, \ I_n(\theta) = \ex{\frac{\partial \ln L(X, \theta)}{\partial \theta}}^2$$
Для смещенных оценок: $\ex{\valuation - \theta}^2 \geq \frac{1+b'(\theta)}{nI_n(\theta)} + b^2(\theta), b(\theta)$ - смещение

\subsection*{Достаточные статистики}
\begin{definition} Т - достаточная статистика, если $$ \prob{X|\theta, T_n} = \prob{X|T_n} $$
\end{definition}

\begin{theorem}
Статистика достаточная, если $p(X, \theta) = g_{\theta}(T_n(X))h(X),\  g_{\theta}(T_n(X))$ не зависит от выборки напрямую \\
	\begin{proof} \\
	$\Rightarrow$ | \\
	$$ \prob{X, \theta, T} = \prob{X, T|\theta}\prob{\theta} = \prob{X|T, \theta}P{\theta}P{T|\theta} = \prob{X|T} \prob{T, \theta} = h(X)g_{\theta}(T_n(X)) $$ \\
	$\Leftarrow$ | \\
	$$ \prob{X|T, \theta} = \frac{\prob{X, T_n, \theta}}{\prob{T_n, \theta}} = \frac{g_{\theta}(T_n(X))h(X)}{\sum_{\theta}(T_n(X'))h(X')} = \frac{h(X)}{\sum h(X)}, \ X'T_n(X') = T_n(X) $$
	\end{proof}
\end{theorem}

\subsection*{Экспоненциальное семейство распределений}

\begin{definition}
$$\prob{X|\theta} = \exp \Big\{ C(\theta) + \sum_{j=1}^p t_j(X)A_j(\theta)\Big\}h(X)$$
$ T = (t_1(X), \ldots, t_p(X)) $ - достаточная статистика
\end{definition}

\begin{theorem}
$A = \braces{x: f(x, \theta) > 0}$ не зависит от параметра $\theta \in \Theta$ и является интервалом на $\setR$ и существует m-мерная достаточная статистика. Тогда: 
\begin{enumerate}
	\item $ m' \leq m \Rightarrow \exists C, \braces{t_j}, \braces{A_j}$ такие что это распределение из экспоненциального семейства
	\item $ m=1 \Rightarrow m'=1$
\end{enumerate}
\end{theorem}

\begin{theorem}[Колмогорова-Блэкуелла-Рао]
Для несмещенной оценки $\valuation: \set{D}\valuation < \infty, \theta \in \Theta$ можно построить $\hat{\theta}_T = \ex{\valuation|T)}$ - несмещенная для достаточной статистики Т и $ \set{D}\hat{\theta}_T \leq \set{D}\valuation $
\end{theorem}

$$ L(X,\theta) = L(\theta) = \log \prob{x|\theta} $$
Оценка методом максимального правдоподобия: $$ \check{\theta} = \arg \max_{\substack{\hat{\theta} \in \Theta}}L(\hat{\theta}) $$

\begin{definition}
Расстояние Кулльбека-Лейблера: $$ D_{KL}(p,q) = \int_{-\infty}^{\infty}p(x) \ln \frac{p(x)}{q(x)} dx $$
\end{definition}

\section*{09.09.15}
$X, \theta$
$$ \prob{\theta|X} = \frac{p(X|\theta)\pi(\theta)}{p(X)} = p(X|\theta)p(theta)$$
\begin{itemize}
	\item $p(X|\theta)$ - правдоподобие
	\item $\pi(\theta)$ - априорное распределение
	\item $p(\theta|X)$ - апостериорное распределение
	\item $p(X)$ - маргинальное распределение, $p(X) = \int_{\Theta} p(X|\theta)\pi(\theta)d\theta$
\end{itemize}

\begin{definition}
$(X_1, \ldots, X_n, \ldots)$ - бесконечно перестановочные, если $\forall n \forall \pi$ $$ p(X_1, \ldots, X_n) = p(X_{\pi_1}, \ldots, X_{\pi_n})$$
\end{definition}

\begin{theorem}[де Финетти]
Пусть последовательность бесконечно перестановочных	$(X_1, \ldots, X_n, \ldots) \Leftrightarrow \forall n$ для некоторой $p$ на $\theta$
$$ p(X_1, \ldots, X_n) = \int \prod_{i=1}^n p(x|\theta) p(d\theta) $$
$p(d\theta) = \pi(\theta)$
\end{theorem}

$$ l(\theta, \delta(x)), \theta \in \Theta, p(x|\theta) $$
$$ \delta(x) = \frac{1}{n} \sum_{\substack{i=1}}^n x_i $$
$$ l(\theta, \delta(x)) = (\theta-\delta(x))^2 $$
Введем понятие риска:
$$ R(\theta, \delta) = \set{E}_{\theta}\squarebraces{l(\theta, \delta(x))} = \int_x l(\theta, \delta(x)) p(x|\theta) dx $$
$\set{E}_{\theta} \delta(x) = \theta$ - несмещенная оценка
$$ \sup_{\substack{\theta in \Theta}} R(\theta, \delta_i) \leq \sup_{\substack{\theta in \Theta}} R(\theta, \delta)$$

\begin{definition}
	Апостериорный риск: $$ \rho(\pi, \delta(x)) = \int l(\theta, \delta(x)) p(\theta|x) d\theta $$
\end{definition}

\begin{definition}
	Байесовское лучшее решающее правило: $$ \delta^*(x) = \arg \min_{\substack{\delta(x)}} \rho(\pi, \delta(x)) $$
\end{definition}

\begin{example}
	$$\rho(\pi, \delta(x)) =\int (\theta - \delta(x))^2 p(\theta|x) d\theta  = \int \squarebraces{(\theta^2p(\theta|x) - 2\theta\delta(x)p(\theta|x) + 
	\delta^2(x)p(\theta|x))d\theta} = 0$$
	$$ \delta^*(x) = \int \theta p(\theta|x)d\theta$$
\end{example}

\begin{definition}
	Байесовский риск: $$ r(\pi, \delta) = \int R(\theta, \delta) \pi(\theta) d\theta = \int p(x, \pi) p(x)dx $$
\end{definition}

\begin{example}
	$$ x_i \sim N(\mu, \sigma^2), \ \mu - ? $$	\\
	\solution{
	$$\pi(\mu) = N(\eta, \tau^2), \eta=100, \tau = \delta^2 $$
	$$ \ex{\mu|X} = \frac{\squarebraces{\frac{\eta}{\tau^2} + \frac{n}{\sigma^2}\bar{x}}}{\squarebraces{\frac{1}{\tau^2} + \frac{n}{\sigma^2}}} = \frac{\eta}{11} + \frac{10}{11}\bar{x}, \ \ \bar{x} = \frac{1}{n}\sum_{\substack{i=1}}^n x_i$$
	}
\end{example}

\begin{definition}
	Сопряженное априорное распределение – априорное и апостериорное распределения лежат в одном семействе
\end{definition}

\begin{example}
	$$ x_i \sim B(1, \theta)$$ $$ \theta \sim \pi(\theta) = \frac{\Gamma(\alpha+\beta)}{\Gamma(\alpha)\Gamma(\beta)}\theta^{\alpha-1}(1-\theta)^{\beta-1} $$
	$$ \Gamma(\theta|X) = c(X) \theta^{\alpha+r-1}(1-\theta)^{\beta + (n-r) - 1} $$
	$$ r = \sum_{\substack{i=1}}^n x_i$$
	$$ \ex{\theta|X} = \frac{\alpha+\beta}{\alpha+\beta+n} \frac{\alpha}{\alpha+\beta} \frac{n}{\alpha+\beta+n} \frac{r}{n} $$
	$\alpha=\beta=1$ - равномерное распределение, $\pi(\theta)=const$ \\
	$\alpha=\beta=\frac{1}{2} \ \Rightarrow \ \pi(\theta) \propto \sqrt{I(\theta)}, \ I(\theta) = -\set{E}_{\theta}\squarebraces{\frac{{\mathrm d}^2 \log p(x|\theta)}{{\mathrm d}^2 \theta}} $
\end{example}

\begin{definition}
	Априорное распределение Джеффри: $$ \pi(\theta) \propto \sqrt{I(\theta)} $$
\end{definition}

\begin{definition}
	Априорное распределение Джеффри - опорное, если $$ KL(\pi(\theta), p(\theta|x)) \to \max $$
	$$ \arg \max MI(\Theta, \setX) = \int \int p(\theta, x) \log \frac{p(\theta, x)}{\pi(\theta) p(x)} d\theta dx $$
\end{definition}

\begin{statement}
	Распределение Джефри инвариантно при трансформации $ \phi=h(\theta)$
	$$ I(\phi) = I(\theta) (\frac{d\theta}{d\phi})^2$$
\end{statement}

\section*{16.09.15}
\begin{definition}
	Регулярность $p(\theta|x)\Pi(\theta), \ \theta \in \Theta$:
	\begin{enumerate}
		\item $\braces{x: p(x|\theta)>0}$ одинаков $\forall \theta \in \Theta$
		\item $L(\theta, x) = \log p(\theta|x)$ трижды дифференцируема по $\theta$ в окрестности $(\theta_0-\delta, \theta_0+\delta)$:
		\begin{itemize}
			\item $\set{E}_{\theta_0} L'(\theta_0), \ \set{E}_{\theta_0} L''(\theta_0)$ - конечны
			\item $\sup_{\theta \in (\theta_0-\delta, \theta_0+\delta)} |L'''(\theta, x)| < M(x), \ \set{E}_{\theta_0}M < \infty$ 
		\end{itemize}
		\item $\set{E}_{\theta_0} L(\theta_0) = 0$, \\ $\set{E}_{\theta_0} L''(\theta_0) = - \set{E}_{\theta_0} (L'(\theta_n))^2$
		\item $I(\theta_0) = \set{E}_{\theta_0} (L'(\theta_0))^2 > 0$
	\end{enumerate}
\end{definition}

\begin{theorem}
	Пусть $p(\theta|x)$ удовлетворяет 1)-4) и $\valuation$ - состоятельная оценка максимального правдоподобия \\
	$\Rightarrow$ $$\frac{1}{\sqrt{n}} (\valuation - \theta_0) \overset{D}{\to} N(0, \frac{1}{I(\theta_0)})$$

	\begin{proof}
		$0 = L'_n(\valuation), \ \valuation$ - ОМП, $L_n(\theta) = \sum_{i=1}^nL(\theta, x_i)$ \\
		$$0 = L'_n(\theta_0) + (\valuation-\theta_0)L''_n(\theta_0) + \frac{(\valuation - \theta_0)^2}{2} L'''_n(\theta'), \ \theta_0<\theta'<\valuation$$
		$$ \sqrt{n}(\valuation-\theta) = \frac{\frac{1}{\sqrt{n}}L'_n(\theta_0)}{-\frac{1}{n}L''_n(\theta_0) - \frac{1}{2}(\valuation-\theta_0)\frac{1}{n}L''(\theta')} $$
		Из ЦПТ:
		$$ \frac{1}{\sqrt{n}}L'_n(\theta_0) \overset{D}{\to} N(0, I(\theta_0)) $$
		Из УЗБЧ:
		$$ -\frac{1}{n} L'_n(\theta_0) - \frac{1}{2}(\valuation-\theta_0)\frac{1}{n}L''_n(\theta') \to i(\theta_0) $$
	\end{proof}
\end{theorem}

\begin{definition}
	Для байесовского случая дополнительные условия:
	\begin{enumerate}
		\item[5.] $\forall \delta > 0 \exists \varepsilon: \set{P}_{\theta_0}\braces{\sup_{|\theta-\theta_0|>\delta} \frac{1}{n}(L_n(\theta) - L_n(\theta_0)) \leq \varepsilon} \to 1$
		\item[6.] $\pi(\theta)$ - априорное распределение - непрерывно и в $\theta_0>0$
	\end{enumerate}
\end{definition}

\begin{theorem}
	Пусть 1)-6) выполнены. Тогда: 
	$$ \int_{\setR} |\pi^*(S|x) - \frac{\sqrt{I(\theta_0)}}{\sqrt{2\pi}}\exp\braces{-\frac{S^2}{2}I(\theta_0)}|ds \overset{\set{P}_{\theta_0}}{\to} 0 $$
\end{theorem}

\begin{theorem}[Бернштейна-фон Мизеса]
	$$ \int \pi(\theta)|\theta|  d\theta < \infty $$
	$$ \theta_n^* = \int_{\setR} \theta p(\theta|x) d\theta $$ - байесовская оценка \\
	Тогда:
	\begin{enumerate}
		\item $ \sqrt{n} (\valuation - \theta_n^*) \overset{\set{P}_{\theta_0}}{\to} 0 $
		\item $ \sqrt{n}(\theta_n^* - \theta_0) \overset{D}{\to} N(0, \frac{1}{I(\theta_0)}$
	\end{enumerate}
\end{theorem}

\subsection*{Априорное распределение Джеффри}
$$ \pi_j(\theta) \propto \sqrt{I(\theta)} $$
$$ I(\theta) = -\set{E}_{\theta_0} \braces{\frac{\partial^2 \log p(x|\theta)}{\partial \theta^2}} $$

\subsection*{Опорное априорное распределение}
$$ \pi^*(\theta) = \arg \max_{\substack{\pi(\theta)}} MI(\Theta, T) $$
T - достаточная статистика $X_1, \ldots, X_n$
$$ MI(\Theta, T) = \int p(t) \int p(\theta, t) \log \frac{p(\theta | t)}{\pi(\theta)} d\theta dt = \int p(t) KL(p(\theta|t)|| \pi(\theta)) dt $$

\begin{example}
	$$ X \sim N(\theta, I), \ x \in \setR^p, \  \ ||\theta||^2 - ? $$
	\solution{
		$$\prob{||theta||^2|X} \sim \Xi^2(p) + ||X||^2$$
		$$ \set{E}_{\theta}\braces{||\theta||^2|X} = ||X||^2 + p $$
		Для опорного распределения Джеффри $$ ||\theta||^2_{prob} $$ - оценка с минимальной дисперсией
		$$ ||\theta||^2_{prob} = ||X||^2-p $$
	}
\end{example}

$T^k$ - набор статистик. $$\pi_k(\theta) = \arg \max_{\substack{\pi(\theta)}} MI(\Theta, T^k)$$
$$ MI(\Theta, T^k) = \int p(t^k) KL(p(\theta|t^k) || \pi(\theta)) dt^k = \int \pi(\theta) \log\frac{f_k(\theta)}{\pi(\theta)}d\theta $$
где $$ f_k(\theta) = \exp \braces{\int p(t^k|\theta) \log p(\theta|t^k) dt^k} $$
Для бесконечного числа экспериментов:
$$ \pi^*(\theta) = \lim_{\substack{k \to \infty}} \pi_k(\theta) $$

\begin{example}
$$ \pi(\theta) \to \braces{p_i}, \ \Theta = \braces{\theta_1, \ldots, \theta_k}, q=f_k(\theta) $$
$$ \arg \max_{\substack{p, \lambda}} \sum p_i \log \frac{q_i}{p_i} + \lambda(\sum p_i-1) - ? $$
\solution{
	Через метод множителей Лагранжа:
	$$ (\log\frac{q_i}{p_i} + p_i \frac{q_i}{p_i}(-\frac{q_i}{p_i^2})) + \lambda = 0 $$
	$$ \log\frac{q_i}{p_i} = \lambda - 1 $$
	$$ \frac{q_i}{p_i} = e^{\lambda-1} \ \Rightarrow p_i=q_ie^{\lambda-1} $$
	$\pi_i(\theta) \sim f_k(\theta) $ - в непрерывном случае
}
\end{example}

\begin{example}
Пусть $\hat{\theta}_k$ удовлетворяет теореме Бернштейна-фон Мизеса:
$$ || p(\theta|\hat{\theta}_k) - N(\hat{\theta}_k, \frac{1}{I_k(\breve{\theta}_n)}) || \to 0 $$
$$ I^{-1}_k(\theta_0) = \frac{1}{k}I^{-1}(\theta_0) $$
$$ p(\theta|\hat{\theta}_k) \propto \sqrt{k} \sqrt{I(\hat{\theta}_k)} \exp \braces{-\frac{I(\hat{\theta}_2)}{2k}(\theta - \hat{\theta}_k)^2} $$
Если $\theta_0$ - истинное значение:
$$  p(\theta|\hat{\theta}_k) \approx \sqrt{k} \sqrt{I(\hat{\theta}_k)} \exp \braces{-\frac{I(\hat{\theta}_2)}{2k}(\theta - \theta_0)^2} = \sqrt{kI(\theta_0)}$$
$\Rightarrow$
$$ f_k(\theta) = \exp \braces{\int p(t^k|\theta) \log \sqrt{I(\theta)} dt^k} = \sqrt{I(\theta_0)} \exp \braces{\int p(t^k|\theta) dt^k} = \sqrt{I(\theta_0)} $$
$\Rightarrow$ в одномерном случае опорное априорное распределение и априорное распределение Джеффри совпадают (но в многомерном - нет)
	
\end{example}

\section*{23.09.15}
\begin{definition}
	$\delta$-сеть: $$ \theta_1, \ldots, \theta_n: \ \forall \theta \in K \exists k: \ \phi(\theta, \theta_k) < \delta $$
	$$ z_{ij} = |T(\theta, x_i) - \mu(\theta) - T(\theta_0, x_1) - \mu(\theta_0)| \to 0 $$
	$$ \lim_{\substack{\delta \to 0}} \sup_{\substack{\theta: |\theta-\theta_n|< \delta}} |T(\theta, x_1) - \mu(\theta) - T(\theta_0, x_1) - \mu(\theta_0)| \to 0$$
\end{definition}

Перепишем условие теоремы:
$$ |\frac{1}{n} \sumi T(\theta, x_i) - \mu(\theta)| = |\frac{1}{n} \sumi \big( T(\theta, x_i) - \mu(\theta)\big)| = |\frac{1}{n} \sumi \big( T(\theta, x_i) - \mu(\theta) - T(\theta_j, x_i) + \mu(\theta_j)\big) + \frac{1}{n} \sumi T(\theta_j, x_i) - \mu(\theta_j)| \leq$$
$$ \leq |\frac{1}{n} \sumi z_{ij}| + |\frac{1}{n} \sumi T(\theta_j, x_i) - \mu(\theta_j)| $$

Хотим: $|\frac{1}{n} \sumi T(\theta_j, x_i) - \mu(\theta_j)| < \epsilon$ для $n>n(\epsilon)$ \\
$\mu(\theta_j) = T(\theta_j, x_1)$ - матожидание при фиквированном x
$$ \delta > 0: |\frac{1}{n} \sumi z_{ij}| < 2\epsilon $$

\begin{theorem}
	$\Theta$ - компактное метрическое пространство, $\theta_0$ - фиксированное
	$$ T(\theta, X) = \log \frac{\set{P}_{\theta}}{\set{P}_{\theta_0}} $$
	Если выполнены условия 1)-2) из предыдущей теоремы:
	\begin{enumerate}
		\item $MLE \ \valuation$ состоятельна в $\theta_0$
		\item $\frac{\prod_{i=1}^n \set{P}_{\theta}(X_i)}{\int_{\Theta} \prod_{i=1}^n \set{P}_{\theta}(X_i)d\Pi(\theta)}$ состоятельна в $\theta_0$
	\end{enumerate}
\end{theorem}

\subsection*{Условия Вальда}
\begin{enumerate}
	\item $\Theta = \bigcup K_i$, где $K_1 \subset K_2 \subset \ldots$ \\
	$\forall \braces{\theta_i}: \theta_i \in K_{i=1} \bigcap K_i$
	$$ lim_{\substack{i \to \infty}} p(X, \theta_i) = 0 $$

	\item $\Phi_i(X) = \sup_{\substack{\theta \in K_{i-1}}} \log \frac{p(X, \theta)}{p(X, \theta_0)} \Rightarrow \set{E}_{\theta_0} \Phi_i(X_1) < \infty$ для некоторого i
\end{enumerate}
$K_1:$ $$ \lim_{\substack{i \to \infty}} \Phi_i(X) = -\infty $$
$K_2: \lim_{\substack{i \to \infty}} \set{E}_{\theta_0} \Phi_i(X_1) = - \infty$

\begin{remark}
	Состоятельность MLE не всегда идет с состоятельностью априорных вероятностных мер
\end{remark}

\begin{example}[Есть состоятельность MLE, нет состоятельности апр. вер. мер] 
	$\Theta = (0,1) \bigcup (2,3), \ p_{\theta}(X)$ - равномерное распределение на $[0, \theta], \ \theta_0 = 1$
	$$ \pi(\theta) = \begin{cases}
		\exp\braces{-\frac{1}{(\theta-\theta_0)^2}}, & \theta \in (0,1) \\
		непрерывная \ ф-цияб & \theta \in (2,3)
	\end{cases} $$
	При заданной выборке $X_1, \ldots, X_n$ (достаточно числа наблюдений и порядковой статистики $X_{(n)}$)
	$\Rightarrow$ $$ \frac{1}{\theta^n} I_{(0, \theta)}(X_{(n)}) = p(\set{X}_n|\theta) $$
	$$ \prob{\theta \in (2,3)} = \frac{\int_2^3 p(\set{X}_n | \theta) \pi(\theta) d\theta}{\Big( \int_0^1 d\theta + \int_2^3 d\theta \Big)} = \frac{\int_2^3 \frac{1}{\theta^n}\pi(\theta)d\theta}{\int_{X_{(n)}}^1 \frac{1}{\theta^n} \pi(\theta)d\theta + \int_2^3 \frac{1}{\theta^n} \pi(\theta) d\theta} \  (*)$$
	Обозначим: $$ I_1 = \int_{X_{(n)}}^1 \frac{1}{\theta^n} \pi(\theta)d\theta  $$ $$ I_2 =  \int_2^3 \frac{1}{\theta^n} \pi(\theta) d\theta)$$
	$I_1) $
	$$ I_1 \leq \pi(X_{(n)}) \int_{X_{(n)}}^1  \theta^{-n} d\theta = \frac{\pi(X_{(n)}}{n-1} \frac{1-X_{(n)}}{X_{(n)}^{n-1}}$$
	$$ \frac{1}{n} \log I_1 \leq \frac{n-1}{n} \log X_{(n)} - \frac{\log(n-1)}{n} + \frac{1}{n} \log (1-X_{(n)}^{(n-1)}) + \frac{1}{n} \log \pi(X_{(n)}) \to -\infty (n \to \infty) $$

	$$ X_{(n)} = \max X_i $$
	$$ \prob{X_{(n)}< 1-\delta} = (1-\delta)^n $$
	$$ \delta = \frac{1}{n}, \ \frac{1}{2} < \alpha < 1 \ \Rightarrow \ \prob{X_{(n)}< 1-\delta} = (1-\frac{1}{n^\alpha})^n$$
	$I_2)$
	$$\frac{1}{3^n} \pi((2,3)) \leq \int_2^3 \frac{1}{\theta^n} \pi(\theta) d\theta \leq \frac{1}{2^n} \pi((2,3)) $$
	$\Rightarrow$ $$ \frac{I_1}{I_2} \to 0, n \to \infty $$
	$\frac{(*)}{I_2} \to 1 \ \Rightarrow $ в окрестности 1 сосредоточена нулевая мера, на (2,3) - единичная вероятность
	$$ \prob{\set{X}_n|\theta} = \frac{1}{\theta^n} I_{(0, \theta)}(X_{(\theta)}) \pi(\theta) $$
	На $(2,3)\bigcup (0,1): \hat{\theta}_{MLE} = X_{(n)}$\\
	При размере выборки, стремящемся к бесконечности, оценка стремится к правдивому значению параметра
\end{example}

Пусть $\Theta = \braces{1, \ldots, k}$ - конечное пространство параметров. Задать в нем распределение = задать вероятностный k-вектор: $p_1, \ldots, p_k: \sum_{i=1}^k p_i = 1 \Leftrightarrow p_1, \ldots, p_{k-1}, 1-\sum_{i=1}^{k-1}$

\begin{example}[Распределение Дирихле]
$$ \prod (p_1, \ldots, p_{k-1}) = \frac{\Gamma(\sum_{i=1}^k \alpha_i)}{\prod_{i=1}^k \Gamma(\alpha_I)} p_1^{\alpha_1-1}  \ldots p_{k-1}^{\alpha_{k-1}-1} \Big( 1- \sum_{i=1}^{k-1} p_i\Big)^{\alpha_k -1} $$
$$ \Gamma(z) = \int_0^{\infty} t^{z-1} e^{-t} dt, \ Re(z) > 0 $$
\end{example}

\begin{theorem}[де Финетти]
	Последовательность случайных величин $x_1, \ldots, x_n$ - перестановочна $\Leftrightarrow \ p(x_1, \ldots, x_n) = \int_{\Theta} \prod_{i=1}^n p(x_i|\theta)dP(\theta) \forall n$ - конечное
\end{theorem}

\begin{example}[урна с шарами]
k типов шаров, $\alpha_i$ - шаров i-го типа.
$\alpha_1, \ldots, \alpha_k:$ $$ \prob{X_2=j|X_1=i} = \frac{\alpha_j + \delta_i(j)}{\sum_{m=1}^k \alpha_m + 1} $$
\end{example}

\subsection*{Методы генерации случайных величин из распределений}
\begin{enumerate}
	\item Из функции распределения: $F_X(x), u \sim U[0,1] \Rightarrow X = F_X^{-1}(u)$
	\item $ x = f(Y), \ p_Y(y)\Rightarrow$ \\
	$$ F_Y(y), \ p_Y(y) = \frac{\partial F_Y(y)}{\partial y}, p_X(x) = \frac{\partial F_X(x)}{\partial x} $$
	$$ \Rightarrow p_X(x)= \frac{\partial F_X(x)}{\partial y} \frac{\partial y}{\partial x} = \frac{\partial F_X(x(y))}{\partial y} \frac{\partial y}{\partial x} \circled{=} $$
	$$ F_X(x(y)) = \prob{X<x} = \prob{X<x(y)} = \prob{f^{-1}(x)<y} = \prob{Y<y} = \begin{cases}
		F_Y(y), & монотонно \ возрастает \\
		1 - F_Y(y), & монотонно \ убывает
	\end{cases} $$
	$$ \circled{=}p_Y(y) sign(f(y)) \frac{1}{f'(y)} = p_Y(y) \frac{1}{|f'(y)|} $$
	\item Box-Bekchen \\
	Круг, вписанный в квадрат с углами в (-1,-1) и (1,1). Кидаем в квардарт точку с координатами $(x_1, x_2)$. Если точка вне круга - выкидываем. Внутри круга - «растягивает» вероятностную меру из каждой точки на луч: $$ y_i = x_i \big( \frac{-2\ln x_i}{r^2}\big)^{\frac{1}{2}} $$
	\item $N(\bar{0}, I_q)$ - генерировать из него $N(\bar{m}, \Sigma)$ \\
	Линейные преобразования: $y=AX+b, \ A \in Mat(q,q), b \in \setR^q$
	$$ \set{E}Y = b $$
	$$ \ex{(Y-b)(Y-b)^T} = \ex{AXX^TA^T} = A \ex{XX^T} A^T = AA^T (из \ предположения \ о \set{E}X)$$
	$ \Rightarrow \ Y = LX + \bar{m} $, где $L:LL^T=\Sigma (\Sigma= U \Lambda U^T - SVD \Rightarrow L=U\Lambda^{1/2} \ или \ L=U \Lambda^{1/2}Q, где \ Q \ ортогональная \ или \ LL^T = LL = \Sigma)$
	\item Методы Монте Карло: метод выборки с отклонением
	\begin{itemize}
		\item умеет считать $\prob{X}$
		\item умеем считать $q(x)$
		\item умеем генерировать $Y \subset q(x)$
		\item надо: $z \subset p(x)$
	\end{itemize}

	$$ k: \forall x \in \set{X} \ kq(x) \geq p(x) $$
	\begin{enumerate}
		\item $x_1 \subset q(x)$
		\item если $u \sim U[0,1]$ и $u < \frac{p(x_1)}{kq(x_1)}$ - принимаем $x_1$, иначе к шагу 1
	\end{enumerate}
	$$ \prob{[x, x+dx]} \approx q(x)dx \frac{\prob{x}}{kq(x)} = \frac{\prob{x}dx}{k}$$
	$$ \breve{\prob{x}} \sim \frac{\prob{x}}{k} $$
	$\Rightarrow$
	$$ \breve{\prob{x}} = \frac{\prob{x}}{k} \frac{1}{\int_\set{X} \frac{\prob{X}}{k} dx} = \prob{x}$$
	$$ \prob{принятие \ с. \ в.} = \int-{\set{X}} \prob{x_1} \frac{1}{kq(x_1)} q(x_1) dx_1 = \frac{1}{k} \int_{\set{X}} \prob{x_1} dx_1 = \frac{1}{k} $$
\end{enumerate}


\section*{30.09.15}
\subsection*{Задача регрессии}
$$ D = (X, \bar{y}) = \braces{(x_i, y_i)_{i=1}^N}, \ x_i \in \setR^p, y_i \in \setR $$ 
$$ f(\bar{x_i}) \sim y_i $$
Линейная регрессия:
$$ f(\bar{x_i}) = \bar{x_i}^T \bar{\theta}, \bar{\theta} \in \setR^p $$
$$  \bar{y} = X\bar{\theta} + \epsilon, \epsilon \sim N(0, \frac{1}{\beta})$$
$$ \ln p(\bar{y}| X, \bar{\theta}) = -\frac{1}{2} \squarebraces{-N \log p + N\log 2\pi + \bar(\hat{y} - X\bar{\theta})^T(\bar{y} - X\bar{\theta})} \underset{\theta}{\to} \max $$
$$  \ln p(\bar{y}|X, \bar{\theta} \underset{\theta}{\to} \max$$
$$ \hat{\bar{\theta}} = (X^TX)^{-1}X^T\bar{y} $$
$$ \frac{1}{\beta_{MAX}} = \frac{1}{N} \sum_{i=1}^N (y_i - \bar{X_i}^T \bar{\theta})^2$$
Если, например, X плохо обусловлена, нужна регрессия

\subsection*{Регуляризация для линейной регрессии}
$$ (\bar{y} - X\bar{\theta})^T(\bar{y} - X\bar{\theta}) + \lambda ||\theta||_q^q \to \min $$
$$ \begin{cases}
(\bar{y} - X\bar{\theta})^T (\bar{y} - X\bar{\theta}) \underset{\theta}{\to} \min, \\
||\theta||_q^q <t
\end{cases} $$
\begin{itemize}
	\item $q=1$ - LASSO
	\item $q=2$ - Ridge
\end{itemize}

$(X^TX) = I$ - X - ортогональная
\begin{itemize}
	\item $q=0$, $\bar{\theta}_{q=0} = \hat{\bar{\theta}}\set{I}\squarebraces{\bar{\hat{\theta}}}>\sqrt{2\lambda} $
	\item $q=1$, $\bar{\theta}_{q=1} = sign(\hat{\bar{\theta}})$
	\item $q=2$, $\bar{\theta}_{q=2} = (|\hat{\bar{\theta}}| - \lambda) + \frac{\bar{\theta}}{1+\lambda}$
\end{itemize}

Если ввести органичение на $\theta$: \\
$$ n(\bar{\theta}) = N(0, \frac{1}{\alpha}I) $$
В этом случае апостериорное распределение $\bar{\theta}$ тоже будет нормально:
$$ p(\bar{\theta}|y,X) \propto N(\beta(\alpha I + \beta X^T)^{-1}X^T\bar{y}, (\alpha I +\beta X^T X)) $$
Обозначим
$$\beta(\alpha I + \beta X^T)^{-1}X^T\bar{y} = \mu_N$$
$$ (\alpha I +\beta X^T X) = S_N $$

$$ p(\bar{y}|X, \alpha, \beta) = \int p(\bar{y}|\bar{\theta}, \beta) p(\bar{\theta}|D, \alpha, \beta) d\bar{\theta} $$
Так как оба распределения нормальные, в итоге получаем нормальное:
$$ p(\bar{y}|X, \alpha, \beta) = N(\bar{X}^T\mu_N, \frac{1}{\beta} + X^T\mu_N X) $$
$$ \ln p(\bar{y}|X, \alpha, \beta) = \frac{p}{2} \ln \alpha + \frac{N}{2}\ln \beta - \frac{\beta}{2}(y-X \mu_N)^T(y - X\mu_N) - \frac{\alpha}{2}\mu_N X^T X \mu_N - \frac{1}{2} \ln |S_N| - \frac{N}{2} \ln 2N$$
Явно решить нельзя, но можно оптимизировать

\subsection*{Логистическая регрессия}
Дана выборка
$$ D = (X, \bar{y}) = \braces{(\bar{x}_i, y_i)}_{i=1}^N, \ x_i \in \setR^p $$
$$ p(y_i=1|X) = 1 - p(y_i=0|X) - ?, \ y_i \in \braces{0,1} $$
$$ p_i = p(y_i=1|X) = \frac{1}{1+\exp{-\bar{x}_i^T\bar{\theta}}} $$
Правдоподобие: 
$$ \log p(y|X, \bar{\theta}) = \log \squarebraces{ \prod_{i=1}^N p_i^{y_i}(1=p_i)^{1-y_i}}$$
$$ \pi(\bar{\theta}) = N(\bar{\theta}_0, \Sigma_0) $$
$$ \log p(\theta|D) \propto -\frac{1}{2}(\hat{\theta} - \theta_0)\Sigma_0^{-1}(\hat{\theta} - \hat{\theta}_0) + \sum_{i=1}^N(y_i) \log p_i + (1-y_i) \log (1-p_i) $$
\subsection*{Аппроксимация Лапласа}
$$ p(\hat{\theta}) \sim q(\hat{\theta}) = N(\arg \max_{\theta \in \Theta} p(\hat{\theta}), \Sigma ) $$
$$ \Sigma^{-1} = \braces{-\frac{\partial^2}{\partial\theta_i \partial\theta_j} \log p(\hat{\theta}^T)} $$
Будем искать аппроксимацию Лапласа для $p(\theta|D)$
$$ \Sigma^{-1} = \Sigma_0^{-1} + \sum_{i=1}^n p_i(1-p_i)\bar{x}_i^T\bar{x}_i $$
$$ \hat{\bar{\theta}} = \bar{\theta}_{MAX} = \arg \max p(\bar{\theta}|D) $$
Не хватает способа получения искомой величины:
$$ p(y_i=1|\bar(X), D) = \int p(y_i=1|\bar{X}, \bar{\theta})p(\bar{\theta}|D) d\bar{\theta} \approx \int \sigma(\bar{\theta}^T \bar{X}) q{\bar{\theta}} d\bar{\theta}, $$
где $$\sigma(\bar{\theta}^T\bar{X}) = \frac{1}{1+\exp{-\bar{\theta}^T\bar{X}}}$$
В таком виде решить не можем, поэтому используем прием: 
$$ \sigma(\bar{\theta}^T\bar{X}) = \int \sigma(a - \bar{\theta}^T\bar{X}) \sigma(a)da $$
$$ p(a) = \int \sigma(a - \bar{\theta}^T\bar{X}) q(\bar{\theta}) d\bar{\theta} $$
$$ p(a) = N(\mu_a, \sigma_a^2), \ \mu_q = \bar{\theta}_{MAX}^T \bar{X} $$
Можно переписать интеграл в таком виде:
$$ p(y_i=1|\bar{X}, D) \approx \int \sigma(a)N(q|_{\mu_q, \sigma_q^2})dq $$
Все еще не можем взять, аппроксимируем еще раз
$$ \sigma(a) \approx \Phi(\lambda a), \ \lambda^2 = \frac{\pi}{8}$$
$$ \Phi(\lambda a) = \int_{-\infty}^{\lambda a} p(x)dx, \ p(x) = N(0,1) $$
$$ \int \Phi(\lambda a)  N(a| \mu_a, \sigma_a^2) da = \Phi(\frac{\mu_a}{(\lambda^{-2} + \sigma^2)^{\frac{1}{2}}}$$

$$ p(y_i=1|\bar{X}, D) =   \Phi(\frac{\mu_a}{(\lambda^{-2} + \sigma^2)^{\frac{1}{2}}},$$
где $\mu_a = \bar{\theta}_{MAX}^T \bar{X}, \ \sigma_a^2 = \bar{X}^T(-\nabla \nabla \log p(\bar{\theta}|D) \bar{X}$

\subsection*{Линейная регрессия(продолжение)}
Априорное распределение Джеффри:
$$ \pi(\frac{1}{\beta}) \propto \beta $$
$$ \bar{\theta} \sim N(\bar{\theta}_0, \frac{g}{\beta}(X^TX)^{-1}) $$
$\Rightarrow$
$$ p(\bar{\theta}|\sigma^2, y, X) \propto N\Big( \frac{g}{g+1}(\frac{\bar{\theta}_0}{g} + \hat{\bar{\theta}}, \frac{1}{\beta} \frac{g}{g+1}(X^TX)^-1)\Big)$$

$$ p(y|X,g) = \frac{\Gamma (\frac{N_1}{2})}{\pi^{\frac{N+1}{2}}\sqrt{N}} || \bar{y} - y_{cp} ||^{-(N_1)} \frac{(1+g)^{\frac{N-1-p}{2}}}{(1+g(1-R^2))^{\frac{N-1}{2}}} $$
$$ y_{cp} = \frac{1}{N} \sum_{i=1}^N y_i, \ R^2 = 1 - \frac{(\bar{y} - X\hat{\bar{\theta}})^T(\bar{y} - X\hat{\bar{\theta}})}{(\bar{y} - y_{cp})^T(\bar{y} - y_{cp})} $$


\section*{07.10.15}
\subsection*{Методы Монте-Карло}
Основаны на двух принципах: марковских сетях и Монте-Карло

\subsection*{Марковские цепи}
\begin{definition}
$ \braces{X_n}_{n \geq 1} $ - марковская цепь $\Leftrightarrow \ \prob{X_n|X_1, \ldots, x_{n-1}} = \prob{X_n|X_{n-1}}$ \\
Задать марковскую цепь - значит, задать \begin{itemize}
	\item $\prob{X_0}$
	\item $T(X_{(m)}, X_{(m+1)}) = \prob{X_{(m+1)}|X_{(m)}}$. Обычно рассматривают однородные марковские цепи, когда такая вер-ть зависит только от наблюдения, а не номера шага
\end{itemize}
На следующем шаге можем записать: $$ \prob{X_{(m+1)}} = \sum_{X_{(m)}} \prob{X_{(m)}} T(X_{(m)}, X_{(m+1)})$$
\end{definition}
\begin{definition}
	Распределение $\pi$ стационарное, если выполнено: $$ \pi(X') = \sum_X \pi(X) \prob{X'|X} $$
\end{definition}
\begin{theorem}[Достаточное условие стационарности (детальное расновесие)]
	$$ \pi(X)T(X, X') = \pi(X')T(X',X)  \ \forall X, X'$$
	\begin{proof}
		$$ \sum_X \pi(x) \prob{X'|X} = \sum_X \pi(X')\prob{X|X'} = \pi(X') \sum_X \prob{X|X'} = \pi(X') $$
	\end{proof}
\end{theorem}

\begin{example}[Условие достаточное, но не необходимое]

	\begin{tabular}{lrrrr}
		{} & A & B & C \\
		A & $\frac{1}{2}$ & $\frac{1}{2}$ & 0 \\
		B & 0 & 0 & 1 \\
		C & 1 & 0 & 0
	\end{tabular}
	$$ \pi = \begin{pmatrix}
		\frac{1}{2} \\ \\
		\frac{1}{4} \\ \\
		\frac{1}{4} \\
	\end{pmatrix}$$ - стационарное распределение.
\end{example}

\begin{definition}
	Марковская цепь называется эргодичной, если $$ \set{P}_{(m)}\braces{X} \underset{m \to \infty}{\to} \pi(X) $$
	Или, что то же самое $ \set{P}^m \to A $: у А все строчки равны
\end{definition}

\begin{definition}
	Марковская цепь регулярна, если она состоит из одного циклического класса (состояния нельзя разбить на одно подмножество так, чтобы стартуя из одного можно было прийти в другое)
\end{definition}

\begin{remark}
	Регулярность $\Leftrightarrow \ \exists m : \set{P}^m$ не имеет нулей
\end{remark}

\begin{lem}
	$ P \in Mat(r\times r), \epsilon $ - ее минимальный элемент, $P$ - вероятностная. $X \in Mat(r \times 1), m_0$ - минимальный элемент, $M_0$ - максимальный элемент. Обозначим $M_1, m_1$ - максимум и минимум для $PX$. Тогда
	\begin{itemize}
		\item $M_1 \leq M_0$
		\item $m_1 \geq m_0$
		\item $M_1 - m_1 \leq (1-2\epsilon)(M_0-m_0)$
	\end{itemize}

	\begin{proof}
		Заменим $X$ на $X'$ такое, что у негов се элементы, кроме одного, равны $M_0$, а один равен $m_0$. Тогда $X' \geq X$ и если $a$ - элемент $P$, который домножается на $m_0$:
		$$ P_{(l)}X' = a m_0 + (1-a) M_0 = M_0 - a(M_0-m_0)$$
		$$ M_1 \leq M_0 - a(M_0-m_0) \leq M_0 - \epsilon(M_0-m_0)$$
		Продолжая рассуждения аналогично: $$ -m_1 \leq -m_0 - \epsilon(m_0-M_0) $$
		Складывая, поучаем искомое
	\end{proof}
\end{lem}

\begin{theorem}
	Регулярные марковские цепи эргодичны \\
	\begin{proof}
		Так как цепь регулярна, то $\exists m: P^m$ без нулевых элементов, и все элементы $\geq \epsilon$. 
		$ e_j = (\delta_{ij}) $ - элемент базиса \\
		$$ P^n e_j = P^{n(1-[\frac{n}{m}]n)}(P^m)^{[\frac{n}{m}]} e_j$$
		Выполнены условия леммы, разность между максимальным и минимальным элементами этого вектора мажорируется $$ (1-2\epsilon)^{[\frac{n}{m}]} \underset{n \to \infty}{\to} $$
		$\Rightarrow$ $$ P^n I \underset{n \to \infty}{\to} A $$
	\end{proof}
\end{theorem}

\begin{example}[Марковская цепь с бесконечным числом состояний]
Пусть множество состояний: $\setR$, начальное состояние $Z_0=0$, а переходная матрица: 
$$ \prob{Z_{(m+1)} = Z_{(m)} + 1|Z_{(m)}} = \frac{1}{4} $$
$$ \prob{Z_{(m+1)} = Z_{(m)} - 1|Z_{(m)}} = \frac{1}{4} $$
$$ \prob{Z_{(m+1)} = Z_{(m)}|Z_{(m)}} = \frac{1}{2} $$
Tак как матрица переходов симметрична и по сути имеем случайное блуждание:
$$\ex{Z_{(m)}} = 0$$ 
Смещение: $\xi_k = Z_{(k)}-Z_{(k-1)}$
$$ \ex{Z_{(m)}^2} = \ex{\sum_{k=1}^m \xi_k^2} = m \ex{\xi_1^2} = \frac{m}{2} $$
$\Rightarrow \set{D}\braces{Z_{(m)}} \sim \sqrt{m}$
\end{example}

\subsection*{Алгоритмы сэмплирования}
\subsubsection*{Metropolis(1953)}
Постановка задачи: Научиться в пределе генерировать с.в. из заданного распределения (умеем считать ненормированную плотность) $$ \check{p}(x): \prob{X} = \frac{\check{p}(x)}{Z_p} $$
$q(X|X^{(\tau)}): q$ - симметрично: $q(X|X') = q(X'|X)$ \\

Алгоритм: \begin{enumerate}
	\item $x^{(0)} = X_0$
	\item $A(x^*, x^{(\tau)}) = \min \big( 1, \frac{\prob{x*}}{\prob{x^{(\tau)}}}\big)$
	\begin{enumerate}
		\item Генерируем $x^*$ из $q(z|z^{(\tau)}$
		\item $x^{(\tau+1)} = x^*$ с вероятнотью $A(x^*, x^{(\tau)}), \ x^{(\tau+1)} = x^{(\tau)}$ иначе
	\end{enumerate}
\end{enumerate}

Из теоремы об эргодичности регулярных МЦ, получаем, что у такого распределения предельное рапределение есть


\subsubsection*{Metropolis-Hastings (1970)}
Условия аналогичные: $\check{p}, q(x|x^{(\tau)}), $ но q не симметрично.
Алгоритм аналогичный, но по-другому определяем $A(x^*, x^{(\tau)})$:
$$ A(x^*, x^{(\tau)}) = \min \big( 1, \frac{\check{p}(x^*) q(x^{(\tau)}|x^*)}{\check{p}(x^{(\tau)}) q(x^*|x^{(\tau)})}\big) $$

Из теоремы об эргодичности регулярных МЦ, получаем, что у такого распределения предельное рапределение есть

$$ p(x) q(x|x')A(x',x) = \min(p(x)q(x|x'), p(x')q(x'|x)) = p(x')q(x'|x) \min(1, p(x)q(x,x')) = p(x')q(x,x')A(x,x') $$


\subsubsection*{Сэмплирование Гиббса(1984)}
Хотим сгенерировать посл-ть из ровно $m$ элементов из некоторого распределения $p(Z) = p(Z_1, \ldots, Z_m$. Умеем для любого $i \prob{Z_i|\set{Z}_{-i}}$ генерировать точку, если есть все распределение, кроме точки i.
\begin{enumerate}
	\item Имеем $Z_1^{(\tau)}, \ldots, Z_m^{(\tau)}$
			$$ Z_1^{(\tau+1)} \sim \prob{Z_1|Z_2^{(\tau)}, \ldots, Z_M^{(\tau)}} $$
			$$ Z_2^{(\tau+1)} \sim \prob{Z_2|Z_1^{(\tau)}, \ldots, Z_M^{(\tau)}} $$
			$$ \ldots $$
			$$ Z_M^{(\tau+1)} \sim \prob{Z_M|Z_1^{(\tau)}, \ldots, Z_{M-1}^{(\tau)}} $$
\end{enumerate}
 Для аналогии с Метрополисом-Хастингом: $q_k(Z^*|Z) = \prob{Z_k^*|\set{Z}_{-k}}$. Покажем, что правило вырождается в тривиальное: 
 $$ A(Z^*, Z) = \frac{p(z^*)q_k(z|Z^*)}{p(z)q_k(z^*|z)} = \frac{ \prob{z^*|\set{Z}_{k-1}^*}\prob{\set{Z}_{-k}^*} \prob{Z_k|\set{Z}_{-k}^*} }{  \prob{z_k|\set{Z}_{k-1}^*}\prob{\set{Z}_{-k}^*} \prob{Z^*|\set{Z}_{-k}^*}  } $$

\begin{remark}
Сложность генерации в этом методе напрямую зависит от сложности генерации $p(Z_k|\set{Z}_{-k})$. Поэтому рассматривают блочные модификации, когда блок переменных зависит не от всех переменных, а только от какогото подблока, для упрощения генерации.
\end{remark}

\begin{remark}
	Эти методы очень удобны в графических моделях, где часто значение $Z_k$ зависит только от своих родителей.
\end{remark}

\begin{remark}
	$$ p(x,y) \propto \exp (-x^2y^2) $$
	$$ Z^{(\tau)} \mapsto Z'^{(\tau)} $$
	$$ Z'^{(\tau)} = \mu_i + \alpha(z^{(\tau)} - \mu^{(\tau)}) + \sigma^{(\tau)}(1-\alpha^2)^{\frac{1}{2}} \nu \sim N(0,I), \ -1<\alpha<1  $$
\end{remark}

\section*{14.10.15}
$p(\theta)$ - вероятностное распределение \\
$$ \int_{\setR^p} f(\bar{\theta}) p(\bar{\theta}) d\bar{\theta} $$

\begin{itemize}
	\item Монте-Карло: $ \int_{\setR^p} f(\bar{\theta}) p(\bar{\theta}) d\bar{\theta} = \sum f(\bar{\theta_i}), \ \bar{\theta} \sim p(\bar{\theta})$
	\item $ q(\theta) \approx p(\theta)$, тогда искомый интеграл берется. Раньше считали: $q(\bar{\theta})$ - нормальное, $\mu = \argmax p(\theta)$, $\sum = \frac{\partial^2}{\partial \theta_i \partial \theta_j} \ln p(\theta)$
\end{itemize}
Будем смотреть, как можно приблизить $q(\theta)$
\subsection*{Вариационный вывод} 
$X$ - данные, $\bar{\theta}$ -параметры модели.
$$ p(X, \theta); \ q(\theta) \approx p(\bar{\theta}|X) - ?; \ p(X) - ? $$
Утверждение: \\
$\ln p(X) = \setL(q) + KL(q|p)$, где 
$$\setL(q) = \int q(\bar{\theta}) \ln \frac{p(X, \bar{\theta})}{q(\bar{\theta})} d\bar{\theta} $$
$$ KL(q|p) = -\int q(\theta) \ln \frac{p(X, \bar{\theta})}{q(\bar{\theta})} d\bar{\theta} $$

$$ \int q(\bar{\theta}) \ln \frac{p(X, \bar{\theta})}{q(\bar{\theta})} d\bar{\theta}  = \int q(\bar{\theta}) \ln p(X) d\bar{\theta} = \ln p(X) $$
Поэтому хорошая стратегия: $KL(q|p) \to \min_q$
Разобьем $\bar{\theta} = (\bar{\theta_1}, \ldots, \bar{\theta_n})$

$q_j(\bar{\theta}_j) = \prod q_j$
$$ \setL(q) = \int \prod q_i \braces{\ln p(X, \bar{\theta}) \sum \ln q_i}d\bar{\theta} = \int q_j \ln p(X, \bar{\theta})\prod_{i \neq j} q_j d\bar{\theta} - \int q_i \ln q_j d\bar{\theta} =$$
$$ = \int q_j \ln \check{p}(\bar{\theta}_j, X)d\bar{\theta}_j - \int q_i \ln q_j d\bar{\theta}_j $$
$$ \ln \check{p}(X, \bar{\theta}_j) = \ex{i \neq j} \ln p(X, \bar{\theta}) + const $$
Подробнее: 
$$ \int \prod q_i \sum \ln q_i d\bar{\theta} = \int \prod q_i \ln q_j d\bar{\theta}_j d\bar{\theta}_i + \int \prod_{i \neq j} q_i \sum_{i \neq j} \ln q_i d\bar{\theta}_j d\bar{\theta}_i$$
Необходимое условие минимума:
$$ \ex{i \neq j} \ln p(X, \bar{\theta}) = \int \ln p(X, \bar{\theta}) \prod_{i \neq j} q_i d\bar{\theta} $$
То есть, мы минимизируем расстояние Кулльбека-Лейблера между $q_j$ и $\check{p}(X, \bar{\theta}_j$. То есть, решаем следующую задачу:
$$ \setL(q) \to \min_{q_j} $$
$$ KL(q_j|\check{p}(X, \bar{\theta})) = - \int q_j \frac{\check{p}(X, \bar{\theta}_j)}{q_j} d\bar{\theta}_j \to \min$$
То есть: 
$$ \ln q_j^*(\bar{\theta}_j) = \ex{i \neq j} \ln p(X, \bar{\theta}) + const $$
$$ q_j^*(\bar{\theta}_j) = \frac{\exp{\ex{i \neq j}{\ln p(X, \bar{\theta})}}}{\int \exp{\ex{i \neq j}{\ln p(X, \bar{\theta}) d\bar{\theta}}}} $$

Алгоритм:
\begin{enumerate}
	\item Выбираем $q_i(\bar{\theta})$
	\item Итерируем $q_i^*(\bar{\theta})$ и ищем $q^* = \frac{q(\bar{\theta})}{q_i(\bar{\theta})}q_i^*(\bar{\theta})$
\end{enumerate}

\begin{example}
	$$p(\bar{\theta}) = N(\bar{\theta}| \mu, \Lambda^{-1})$$
	$$ \mu = (\mu_1, \mu_2), \Lambda = ( \lambda_{11}, \lambda_{12}, \\ \lambda_{21}, \lambda_{22}) $$
	$$ q(\bar{\theta}) = q_1(\bar{\theta}_1)q_2(\bar{\theta}_2) $$
	$$ \ln q_1*(\theta_1) = \ex{\theta_2} \ln p(\bar{\theta}) + const  = \ex{\theta}{-\frac{1}{2}(\theta_1 - \mu_1)^2 \Lambda_{11} - (\theta_1-\mu_N)\Lambda_{12}(\theta_2-\mu_2)} + const =$$  

	$$= -\frac{1}{2} \theta_1^2 \Lambda_{11} + \theta_1 \mu_1 \Lambda_{11} - \theta_1 \Lambda_{12}(\ex{theta_2}-\mu_2) + const $$

	$$ q_1^*(\theta_1) = N(\theta_1|m_1, \Lambda_{11}^{-1}) $$

	$$ m = \mu_1 - \Lambda_{11}^{-1}\Lambda_{12}(\ex{\theta_2} - \mu_2) $$
	$$ q_2^*(\theta_2) = N(\theta_2|m_2, \Lambda_{22}^{-1}) $$
	$$ m_2 = \mu_2 - \Lambda_{11}^{-1}\Lambda_{12}(\ex{\theta_1} - \mu_1) $$
	$$ \begin{cases}
		m_1 = \mu_1 - \Lambda_{11}^{-1}\Lambda_{12}\squarebraces{m_2-\mu_2} & \\
		m_2 = \mu_2 - \Lambda_{22}^{-1}\Lambda_{21}\squarebraces{m_1-\mu_1} \\
	\end{cases} $$
	$$ m_1 = \mu_1, \ m_2 = \mu_2 $$

	$$ KL(p||q) \to \min_q, \ KL(p|q) = -\int p(\bar{\theta}) \sum_{i=j}^m \ln q_i(\bar{\theta}) d\theta$$
	$$ q_i^*(\bar{\theta}) = \int p(\bar{\theta}) \prod_{j \neq i} d\bar{\theta}_j = p(\bar{\theta}_j) $$

	$$ KL(q|p) = -\int q(\theta) \ln \frac{p(\theta)}{q(\theta)} d\theta $$
	$$ KL(p|q) = - \int p(\theta) \ln \frac{q(\theta)}{p(\theta)} $$

\end{example}

Можно описать более строго, если ввести семейство распределений:
\begin{definition}
$$ D_{\alpha}(p|q) = \frac{1}{1-\alpha} (1-\int p(x)^{(1+\alpha)/2})q(x)^{(1-\alpha)/2} dx )$$
\begin{enumerate}
	\item $\alpha \to 1 \Rightarrow KL(p|q)$
	\item $ $
	\item $D_{\alpha}(p|q) \geq 0$
	\item $D_{\alpha}(p|q) = 0 \Leftrightarrow p(x) = q(x)$
	\item $$ \alpha \leq -1 \Rightarrow D_{\alpha}(p|q) \to \min_q, \ p(x) = 0 \Rightarrow q(x)=0 $$
	\item $\alpha \geq 1 \Rightarrow D_{\alpha}(p|q) \to \min_q \Rightarrow p(x) >0, q(x) > 0$
	\item ...
\end{enumerate}
\end{definition}

Найдем расстояние Кулльбека-Лейблера между произвольным распределением и распределением из экспоненциального семейства: 
$$ KL(p|q) = -\ln q(\bar{\eta}) - \eta^T \ex{p(\bar{\theta})}{u(\bar{\theta})} + const $$
$$ - \nabla \ln g(\bar{\eta}) = \ex{p(\bar{\theta})} u(\bar{\theta}) $$
$$ \ex{q(\bar{\theta})}{u(\bar{\theta})} = - \nabla \ln g(\bar{\eta}) $$
Из метода моментов:  $$ \ex{q(\bar{\theta})}{u(\bar{\theta})} = \ex{p(\bar{\theta})}{u(\bar{\theta})} $$

Дл минимизации этого расстояния предполагаем, что правдоподобие имеет специальный вид: 
$$ p(X, \bar{\theta}) = \prod_i f_i(\bar{\theta}) $$
$$ f_i(\bar{\theta}) = p(\bar{X}_i|\bar{\theta}) $$
$f_0(\bar{\theta}) = \pi(\bar{\theta})$ - априорная факторизация
$$p(\theta|X) - ? \ \ p(X) - ?$$
Запишем в явном виде для апостериорного распределения
$$ p(\bar{\theta}|X) = \frac{1}{p(X)} \prod f_i(\bar{\theta}) $$
$$ p(X) = \int \prod_i f_i(\bar{\theta})d\bar{\theta} $$
Можем приблизить $f_i \to \check{f_i}$. Предполагаем \begin{enumerate}
	\item $q(\bar{\theta}) = \frac{1}{z} \prod_i \check{f}_i(\bar{\theta}), \ \check{f}_i(\bar{\theta}) \sim f_i(\bar{\theta})$
	\item $\check{f}_i(\bar{\theta})$ - из экспоненциального семейства
\end{enumerate}
Варианты решения оптимизационной задачи: \begin{enumerate}
	\item $ KL(f_i(\theta)|\check{f}_i(\theta)) \to \min$
	\item $ KL(p|q) = KL(\frac{1}{p(X)}\prod_i f_i(\bar{\theta}))| \frac{1}{z} \prod \check{f}_i(\bar{\theta})$
	\item По очереди оптимизируем пред. выражение по $\check{f}_i(\bar{\theta})$
\end{enumerate}

\begin{enumerate}
	\item $q(\bar{\theta}) = \frac{1}{z} \prod_i \check{f}_i(\bar{\theta})$
	\item $ q^{min}(\bar{\theta}) \propto \check{f}_j(\bar{\theta}) \prod_{i \neq j} \check{f}_j(\bar{\theta}) $ - близка к $f_j(\bar{\theta})\prod \check{f}_j(\bar{\theta}), \ q^j(\bar{\theta}) = \frac{q(\bar{\theta})}{\check{f}_j(\bar{\theta})}$
\end{enumerate}
\begin{example}
	$$p(\bar{X}|\bar{\theta}) = (1-\omega) N(\bar{X}|\bar{\theta},I) + \omega N(\bar{X}|0,aI)$$
	$\omega, a$ - известны. Априорное распределение:
	$$ \pi(\bar{\theta}) = N(\bar{\theta}|0,bI) $$

	$$ p(X, \bar{\theta}) = \pi(\bar{\theta}) \prod_{n=1}^N p(\bar{X}_n|\bar{\theta}) $$
	$$ f_0(\bar{\theta}) = \pi(\bar{\theta}) $$
	$$ v_n^{-1} = (v^{new})^{-1} - (v^{\\n})^{-1} $$
	$$ q^n(\bar{\theta}) = N(\bar{m}^{\\n}, (v^{\\n})^{-1}I$$
	$$ \bar{m}^n  = \bar{m} + \frac{v^{\alpha_n}}{v_n} (\bar{m} - \bar{m}_n$$
	$$ q^{new}(\bar{\theta}) = q^{\\n}(\bar{\theta})f_n(\bar{\theta}) $$
	$$ \check{f}_n(\bar{\theta}) = s_n N(\theta| \bar{m}_n, v_nI) $$
	$$ s_n = \frac{z_n}{(2\pi v_n)^{p/2}N(\bar{m}_n| \bar{m}^{\\n}, (v_n + v^{\\n})I} $$
\end{example}



\section*{21.10.2015}
\begin{remark}
	Презентацию можно взять у Бурнаева
\end{remark}

\section*{28.10.15}
До сих пор было $\pi(\theta)$ - априорное распределение и $\braces{P_\theta, \theta \in \Theta}$
\subsection*{Байесовскийесовское задание гладкости}
Задается ковариационная функция $K(X, X')$, если $K(X, X') \to 0, X' \to X$, то процесс «гладкий» (???). У нас процесс гауссовский, решаем задачу регрессии $y_i = f(x_i) + \epsilon_i$, предполагаем, что $f$ - реализация гауссовского поля, $\epsilon$ - белый шум. \\
Пусть есть какой-то параметрический класс ковариационных функций (у нас это $\sim exp\braces{-dist(X, X')^2}$).
Существует теорема о нормальной корреляции:
$$ \begin{array}{c} y_1 \\ y_2 \end{array} \in N 
\begin{pmatrix} 
\mu_1 & \Sigma_{11} & \Sigma_{12} \\ 
\mu_2 & \Sigma_{21} & \Sigma_{22}	
\end{pmatrix}$$
$$ \Rightarrow Law(y_2|y_1) = N(\mu_1 + \Sigma_{21} \Sigma_{11}^{-1} (y_1-\mu_1), \Sigma_{22} - \Sigma_{21}\Sigma_{11}^{-1}\Sigma_{12})  $$

$$  Law(y(x)|y(x_1), \ldots, y(x_n)) \sim N(\hat{f}(x), \hat{\sigma}^2(x)) $$
$$ \hat{f}(x) = \sum \alpha_i K(x, x_i) $$
$$ \hat{f}(x) = k(x) K y \ \ (k(x) = K(x,x), \ K=K(x_i, x_j)_{i,j})$$
В отличие от ядерной регрессии, можно оценить дисперсию и построить доверительное множество
$$ \hat{\sigma}^2(x) = K(x,x) - k(x)K^{-1}k(x)^T $$
Оценка параметров ковариационной функции - максимальное правдоподобие

$$K(x,x') = \sigma^2 K(x,x'|\theta)$$
Из белого шума:
$$ K(x,x') = K(x,x') + \sigma \delta(x,x') $$

$K = K + \sigma^2I$ - имеет свойство регуляризации
$$ K = \sigma^2 K_{\theta}(x_i, x_j) $$

\section*{11.11.15}
$$ X \in \setR, p(x|\theta), \hat{\theta} \in \Theta, \pi(\hat{\theta}) $$
Какое может быть семейство распределений на $x=\braces{1,2}$? Все элементы эквиваленты точке в области $\braces{\hat{p}=(p_1,p_2), p_1, p_2 \geq 0, p_1+p_2=1}, \ p_1=\prob{x=1}, p_2=\prob{x=2}$. Надо ввести априорное распределение на $p_1$ и $p_2$. Пример: бета-распределение
\begin{definition}
	С.в. распределена с $\sim \beta$-распределением с пар-рами $\alpha_1, \alpha_2 > 0$, если $$ \prod (p_1) = \frac{\Gamma(\alpha_1 + \alpha_2)}{\Gamma(\alpha_1)\Gamma(\alpha_2)} p_1^{\alpha_1-1} (1-p_1)^{\alpha_2-1}, \ 0 \leq p_1 \leq 1$$
	Если $\alpha_1 \to 0 \Rightarrow \prob{p_2=1}=1$
\end{definition}

Пусть есть нез. с.в. $z_1, z_2$ с распр. $\Gamma$ с пар-рами $\alpha_1, \alpha_2$, т.е. $p(z_i) = \frac{1}{\Gamma(\alpha_1)} \exp{-z_i}z_i^{\alpha_i-1}, \ z_i > 0$
$$ \frac{z_1}{z_1+z_2} \sim \beta(\alpha_1, \alpha_2) $$
$$ x_1, \ldots, x_n \in \braces{1,2}, \hat{p} = (p_1, p_2)$$
$\Rightarrow \beta(\alpha_1, \alpha_2)$ - сопряженное в том смысле, что если у нас есть $p(\hat{p}|\hat{x}) = \beta(\alpha_1',\alpha_2')$, где 
$$ \alpha_1' = \alpha_1 + \sum \squarebraces{x_i == 1}, \ \alpha_2' = \alpha_2 + \sum \squarebraces{x_i == 2}$$
Рассмотрим более общий случай: $x_i \in \setX = \braces{1, \ldots, k}$. Все распределения = множеству векторов $\braces{\hat{p} = (p_1, \ldots, p_k), p_i \geq 0, \sum p_i = 1}, \ \prob{x=i}=p_i$
\begin{definition} Распределение Дирихле \\
	$\bar{\alpha} = \braces{\alpha_1, \ldots, \alpha_k}, \alpha_i \geq 0$
	$$ \prob (p_1, \ldots, p_{k-1}) = \frac{\prod(\sum \alpha_i)}{\prod_{i=1}^k \Gamma(\alpha_i)}\prod_{i=1}^k p_i^{\alpha_i-1} (1-\sum_{i=1}^{k-1} p_i)^{\alpha_k-1}$$
\end{definition}

Свойства:
\begin{enumerate}
	\item $z_1, \ldots, z_k, \ z_i \sim \Gamma(\alpha_i)$ - независимые. Тогда вектор $\big( \frac{z_1}{\sum z_i}, \ldots \frac{z_k}{\sum z_i} \big) \sim D(\hat{\alpha})$
	\item $A_1, \ldots, A_n \subset \setX, A_i пересеч. A_j = непуст., UA_i = \setX=\braces{1, \ldots, k} \Rightarrow \big( P(A_1), \ldots, P(A_m) \big) \sim D(\alpha_1', \ldots, \alpha_m'), \ \alpha_i' = \sum_{j \in A_j} \alpha_j$ \\
	\item Удобно говорить о $\alpha(A)$ как о мере: $ \alpha(A) = \sum_{i \in A_i} \alpha_i$. Тогда можно переписать: $$ \sim D(\alpha(A_1), \ldots, \alpha(A_m)) $$
	\item $\prob(X_{n+1}=k|X_1, \ldots, X_n) = \frac{\alpha_k + \sum_{i=1}^n [x=k]}{\alpha(\setX)+n}, \ p(p) = D(\hat{\alpha})$
	$$ p(\hat{p}|\hat{x}) = D(\hat{\alpha}'), \alpha_i' = \alpha_i + \sum_{j=1}^n [x_j=i] $$
\end{enumerate}

Пример с шарами: 
$$ \prob{x_1=j} = \frac{\alpha(j)}{\sum_{i=1}^n \alpha(i)} $$
$$ \prob{x_2=j|x_1} = \frac{\alpha(j) + [x_1=j]}{\alpha(\setX) + 1} $$
$$ \prob{x_1=X_1, \ldots, x_n=X_n} = \frac{\alpha(X_1)}{\alpha(\setX)} \Big( \frac{\alpha(X_2) + [x_1=x_2]}{\alpha(\setX)+1}\Big) \ldots 
\frac{\alpha(X_n) + \sum_{i=1}^{n-1}[x_i=x_n]}{\alpha(\setX)+n-1} $$

Можно расширить на случай $\setX=\setR^d, \ M(\setX)$. Вводим в качестве априорного распределения распределение Дирихле. Если задана некоторая мера $H$ на множестве $\setR^d$ и некоторое $\alpha>0$, то $\exists !$ вероятностная мера  $D_{\alpha, H}$ на $M(\setR^d)$ т.ч. при любом разбиении $\setR^d$ на борелевские мн-ва $B_1, \ldots, B_m$:
$$ (P(B_1), \ldots, P(B_m)) \sim D(\alpha H(B_1), \ldots, \alpha H(B_m)) $$, если $\alpha>0, H$ - конечная мера. Тогда это случайный процесс Дирихле

\subsection*{Сравнение гауссовского процесса и процесса Дирихле}
Для гауссовского процесса берем $x \in \setR^d$. Для процесса Дирихле $A \in B(\setR^d)$.
Реализация процесса:
1)$ f: \setR^d \to \setR $, 
2)$ G$ - вероятностная мера
Параметры процесса:
1) среднее $\mu(\hat{x})$, ковариационную функцию $k(\hat{x}, \hat{x}')$. 
2) Задать вер. меру $H$ на $\setR^d$ и некоторое $\alpha>0$
Одномерная проекция:
1) $f(\hat{x}) \sim N(\mu(\hat{x}), k(\hat{x}, \hat{x}'))$ 
2) $G(A) \sim \beta(\alpha H(A), \alpha (1-H(A)))$
Многомерный срез:
1) $f(\hat{x}_1), \ldots, f(\hat{x}_n) \sim N((\mu(\hat{x_1}), \ldots \mu(\hat{x_n})), K)$ 
2)$A_1, \ldots A_n$. $G(A_1) ,\ldots, G(A_n) \sim D(\alpha H(A_1), \ldots, \alpha H(A_n))$


Важно!
\begin{enumerate}
	\item С вероятностью 1 реализация процесса Дирихле – это дискретная вероятностная мера: $p(x) = \sum_{i=1}^\infty p_i \delta_{x_i}(x), \ p_i >0, \sum p_i=1$
	\item Рассмотрим модель: $G \sim D(\alpha, H)$. Генерируем из вер. меры набор $x_1, \ldots x_n | G \sim G$. Можем явно выписать апостериорное распределение: $G | x_1, \ldots x_n \sim D(\frac{\alpha}{\alpha+n}H + \frac{1}{\alpha+n}\sumi \delta_{x_i}, \alpha+n)$
	$$ x_n+1 | x_1 \ldots, x_n \sim \frac{\alpha}{\alpha+n}H + \frac{1}{\alpha+n}\sumi \delta_{x_i}$$
\end{enumerate}


\begin{definition}[Процесс «китайский ресторан»]
	$x_1, \ldots x_n$ - уже есть. $x_{n+1}|x_1, \ldots, x_n \sim H$ с вер-тью $\frac{\alpha}{\alpha+n}$, $\delta_c$ с вероятностью $\frac{\sum[x_j=c]}{\alpha+n}$
\end{definition}
Получаем условно кластеры. Можем посчитать матожидание их числа. Пусть на шаге $n$ число кластеров - $m$.
$$ \setE_{\alpha, n}m = \alpha(\psi(\alpha+n) + \psi(\alpha)) = \alpha \log (1 + \frac{n}{\alpha}), n \to \infty $$

\subsection*{Смесь распределений с априорным распределением, заданным процессом Дирихле}
Смесь распределений: $p(\hat{x}|\hat{\theta}), \hat{\theta} \in \setR^p$. $\hat{\theta}$ - априорн. распределение G, G - реализация процесса Дирихле.
$$ G \sim D(H, \alpha), \theta_i \sim G, \ x_i \sim p(x_i|\theta_i), \ i=1,\ldots,n $$. 
Модельный пример: 
\begin{example}
	Пусть $p(x|\theta) = N(\hat{\mu}, \sigma_X I), \hat{\mu} = \theta, \sigma_x$ - известно. \\
	$p_H(\theta) = N(0, \sigma_{\mu}), \sigma_{\mu}$ - известно. Тогда
	$$ p(\mu|x_1, \ldots, x_n) = N(\frac{\sigma_n}{\sigma_X} \mu, X_i, \sigma_n), \frac{1}{\sigma_n} = \frac{1}{\sigma_{\mu}} + \frac{1}{\sigma_X} $$
	$$ p(x) = \int p(x|\mu)p_H(\mu)d\mu = N(0, \sigma_{\mu} + \sigma_X) $$
\end{example}
Проведем сэмплирование Гиббса:
$\braces{\hat{\theta_i}}, i=1, \ldots, n$. $$ p(\theta_i | \hat{\theta}_{\\i}, \hat{x}), \hat{\theta}_{\\i} = braces{\hat{\theta_j}}_{j \neq i} $$

$$ p(\theta_i | \hat{\theta}_{\\i}), \hat{x}) = \frac{p(\theta_i | \hat{\theta}_{\\i})) p(x_i|\theta_i)}{\int p(\theta_i | \hat{\theta}_{\\i})) p(x_i|\theta_i) d\theta_i} $$
$$ p(\theta_i | \hat{\theta}_{\\i})) = \frac{\alpha}{\alpha+n} p_h(\theta_i) + \frac{1}{\alpha+n-1} \sum_{i\neq j} \delta_{\theta_j}(\theta_i) $$
$$ p(\theta_i | \hat{\theta}_{\\i}), \hat{x}) = \frac{1}{z} (\alpha (\int p(x_i||\theta)p_H(\theta) d\theta) p(\hat{\theta}_i|x_i)) +
\sum_{j \neq i} \delta_{\theta_j}(\theta)p(x_i|\theta_j))$$
$$ p(\mu_i|\hat{\mu_{x_i}}, \bar{x}) = \frac{1}{z} \alpha N(x_|0, (z_{\mu} + z_x)) N(\mu_i|\frac{\sigma_1}{\sigma_x}x_i, \sigma_1) + 
\frac{1}{z}  \sum \delta_{\mu_j}(\mu)N(x_i|\mu_j, \sigma_x)$$

Генерировать распределение Дирихле можно еще одним способом: «ломание палки»
$$ p(\bar{p}) = p(p_1)p(p_2|p_1)p(p_3|p_2, p_1) $$
$$ p_1 = v_1 \sim \beta(\alpha_1, \sum_{i=2} \alpha_i), v_2 \sim \beta(\alpha_2, \sum_{i=3} \alpha_i) ... $$
$$ p_k = 1 - \sum_{i=1}^k p_i, p_i = v_i \prod_{j=1}^{i-1}(1-v_j) $$


\section{18.11.15}
Проверяем следующие свойства для распределения Дирихле:
\begin{enumerate}
	\item Из накопления $Dir(\alpha_1, \ldots, \alpha_n, \alpha_{n+1})$
	\item из $\alpha(B_n) \to \alpha(B)$ и свойства выполняются
\end{enumerate}

\begin{itemize}
	\item Индикаторный элемент: Гаус. проц: $x \in \setR^d$, Проц. Дир: $A \in \tau \subset 2^{\setR^d}$
	\item Реализация: $f: \setR^d \to \setR^q$ и вероятностное распределение $ G: \tau  \to [0,1]$
	\item Параметр $ \mu(x), K(x, x)$ и опорное вер. распр. $H$ и концентрация $\alpha>0$
	\item Одномерное сечение $Y(x) \propto N(\mu(x), K(x,x))$ и $A: \xi(., A) \sim \beta(\alpha H(A), \alpha (1-H(A)))$
	\item Многомерное сечение $\bar{y} = (y(x_1), \ldots, y(x_n)) \sim N(\bar{\mu}, K)$ и $ A_1, \ldots, A_n, \ P(A_1), \ldots P(A_n) \sim Dir(\alpha(H(A_1), \ldots \alpha H(A_n))) $
\end{itemize}

Процесс Дирихле дискретен: $$ P(x) = \sum_{i=1}^\infty p_i \delta_{x_i} $$
Условное распределение: $$ G \sim Dir(H, \alpha), \ \theta_1, \ldots \theta_n | G \sim G $$
$$ G | \theta_1, \ldots, \theta_n \sim Dir(H \frac{\alpha}{\alpha + n} + \frac{\sum_{i=1}^\infty \delta_{\theta_i}}{\alpha+n}, \alpha+n) $$
$$ \theta_{n+1} | \theta_1, \ldots, \theta_n \sim \frac{\alpha}{\alpha+n}H + \frac{1}{\alpha+n} \sum_{i=1}^\infty \delta_{\theta_i}$$


$$ O(log n) \to O(n^d), d \in [0,1] $$

$$ \theta_{n+1} | \theta_1, \ldots, \theta_n \sim H \subset \frac{\alpha\mu(\theta_1, \ldots, \theta_n)} , \delta_{\theta} \subset \frac{(\sum_{i=1}^\infty[\theta_i = \theta]) - d}{\alpha+n}$$

В иерархическом процессе Дирихле тематического моделирования объекты – слова, группы объектов - документы, кластемы - темы
$$ G_0 | H, \alpha \sim Dir(H, \alpha) $$
$$ G_j | G_0, \alpha_0 \sim Dir(G_0, \alpha_0), j=1, \ldots, \infty $$

Процесс Дирихле для смеси:
$$ G \sim Dir(H, \alpha), (\theta_i|G) \sim G, \ x_i \sim p(x_i|\theta_i) $$
Смесь гауссиан:
$$ \Sigma = const = I $$
$$ \bar{\theta} = (\bar{m}, \sigma): p(x_i|\theta) \sim N(\bar{\mu}, \sigma^2I) $$



\section{25.11.15} 
Необходимые свойства:
\begin{itemize}
	\item $Dir(p|\alpha) = \frac{\Gamma(\sum \alpha_i)}{\prod \Gamma(\alpha_i)}\prod p^{\alpha_i-1}$
	\item $(\sum_{i \in A_1} p_i, \ldots \sum_{i \in A_k} p_i) \sim Dir(\hat{p}|\sum_{i \in A_1} \alpha_i, \ldots \sum_{i \in A_k} \alpha_i)$
	\item $p \sim Dir(\alpha), p_i \indep \frac{1}{1-p_i}(p_1 \ldots p_{i-1}, p_{i+1}, \ldots p_d)$
\end{itemize}

Генерация из $c_i \sim Gam(\alpha_i, 1), \ p_i=\frac{c_i}{\sum c_k}, i=1, \ldots, d$
\subsection{Способы генерации}
\subsubsection{Процесс поломки палки} 
Идея: есть $\alpha_1, \ldots, \alpha_d$. Нужно сгенерировать распределение с такими параметрами.
\begin{enumerate}
	\item $p_1=v_1 \sim Beta(\alpha_1, \sum_{j=2}^d \alpha_j)$
	\item $p_2=v_2(1-v_1), v_2 \sim Beta(\alpha_2, \sum_{j=3}^d \alpha_j)$  \\
	...
	\item $p_i = v_i \prod_{j=1}^{i-1}(1-v_j), v_j \sim Beta(\alpha_i, \sum_{j=i+1}^d \alpha_j)$
\end{enumerate}
Для того, чтобы оборвать процесс: $v_d=1, \ p_d=1-\sum_{i=1}^{d-1}p_i$

\subsection{Модельные наблюдения и кластеризация}
Пусть модель генерируется из $x_i \sim N(\mu, \sigma_x I)$ - кластер. Будем говорить, что $\mu \sim N(0, \sigma_{\mu} I)$
$$ p(\mu|x_1, \ldots, x_n) = N(\frac{\sigma_N}{\sigma_x}\sum x_i, \sigma_n I), \frac{1}{\sigma_n} = \frac{1}{\sigma_{\mu}} + \frac{n}{\sigma_x} $$
$$ G \sim DP(H, \alpha), H = N(0, \sigma_{\mu}I) $$

Предполагаем $\mu \sim G$ и тогда генерируем $x_i \sim N(\mu_i, \sigma_x I)$. Из идеи китайского ресторана:
$$ \mu_1\ \ldots \mu_n \Rightarrow \mu_i|\mu_1 \ldots \mu_{i-1} \sim 1) (H, \frac{\alpha}{\alpha_n+i-1}), \ 2)\mu_c, \frac{\sum [\mu_k=c]}{\alpha+h}  $$

Как организовать процесс генерации точек с помощью одного из методов Монте-Карло?
Способы оценки процесса Дирихле:

\subsubsection{MCMC (Monte-Karlo, Markov chain)}
Пусть можем генерировать каким-то образом $X = (X_1, \ldots, X_n)$, а вектор $\theta_i$ - все параметры, которые нам интересны. Допустим, что вер модель устроена так, что условные вер-ти .$p(\theta_i|\theta_{-i}, X)$. можем подсчитать аналитически или имеет алгоритм, позволяющий генерировать из вероятностей. Тогда итерируем по i, генерируем $\theta$, тогда получим $p(\theta|X)$. Вопрос, как подсчитать
$$ p(\theta_i, X_i | \theta_{-i}, X_{-i}) = p(\theta_i|\theta{-i})p(x_i | \theta_i) $$
Если параметр фиксирован, то $X$ генерируются из соответствующих распределений 
$$ p(\theta_i | \theta_{-i}, X_i) = \frac{p(\theta_i|\theta_{-i}) p(X_i|\theta_i) }{\int p(\theta_i|\theta_{-i}) p(X_i|\theta_i) d\theta_i} $$
$$ p(\theta_i | \theta_{-i}) = \frac{\alpha}{\alpha+n-1} p(\theta_i) + \frac{1}{\alpha+n-1} \sum_{j \neq i} \delta_{\theta_j}(\theta_i) $$
Каждое наблюдение генерируется из своего кластера с параметром $\theta_i$. Хотим оценить совместное распределение $\theta_1, \ldots, \theta_n$ с учетом того, что некоторые могут совпадать. 

Для нормального распределения:
$$ p(\mu_i|\mu_{-i}, X) = \frac{\alpha}{Z} N(X_i|0, (\sigma_x + \sigma_{\mu})I) N(\mu_i | \frac{\sigma_1}{\sigma_x}x_i, \sigma_1 I) + \frac{1}{Z} \sum_{k \neq i}^n \delta_{\mu_j}(\mu_i)N(\mu_i | \mu_j, \sigma_{\lambda}I) $$
$$ Z = \alpha N(X_i|0, (\sigma_x + \sigma_{\mu})I) + \sum_{j \neq i} N(X_i|\mu_j, \sigma_xI) $$
Этот метод медленно сходится, поэтому:

\subsubsection{MacEachern scheme}
Пусть есть переменная $Z_i \in N$ - указывает кластер, $\theta_k$ - описывает k-ый кластер, и есть наблюдение $X_i$. Ненаблюдаемые переменные $(Z_1, \ldots Z_n) \sim CRP(\alpha, n)$ - порoждаются процессом китайского ресторана \\
$p(z_i| z_{\-i}) = 1) (\frac{\sum_{j \neq i} I[z_j=c]}{\alpha+n-1})$, если $z_i$ принадлежит кластеру $c$, а если $z_i$ образует новый кластер: $ \frac{\alpha}{\alpha+n-1}$.
$$ \theta_i \sim H, \ X_i \sim p(x_{-i} | \theta_i) $$
Дальше хотим генерировать наблюдения из распределения $\bX = (X_1, \ldots, X_n), \ p(Z, \theta| \bX)$ с плотностями, $p(z_i|z_{-i}, \theta, \bX), \ p(\theta_k | z, \theta_{-k}, \bX)$
Метод сходится гораздо быстрее и сразу дает индикаторы принадлежности к кластерам

При фиксированном $Z$:
$$ p(X, \theta | Z) = \prod_{k=1}^K p_H(\theta_k) \prod_{i=1}^n p(x_i|\theta_{Z_i}) $$

$$ p(\theta_k|\theta_{-i}, Z, X) \propto p_H(\theta_i) \prod_{i:z_i=k}p(x_i|\theta_k) $$
$$ p(z_i=k|Z_{-i}, \theta, X) = \frac{n_{-i,k}}{z_i (\alpha+n-1)}p(x_i|\theta_k) $$

Вероятность попасть в существующий кластер:
$$ h_{-i,k} = \sum_{j \neq i} I[z_j=k] $$
Если создается новый кластер сначала нужно сгенерировать для него $\theta_{new}$:
$$ p(z_i=new, \theta_{new} | Z_{-i}, \theta, X) = \frac{\alpha}{Z_i (\alpha+n-1)} p(x_i|\theta_{new})p_H{\theta_{new}} $$

$$p(z_i=new|... = \int p(z_i=new, \theta_{new}| ...)p_H(\theta_{new}) d\theta_{new} $$


\subsubsection{Коллапсированная схема Мак-Ичерна}

$$ p(z_i=k | Z_{-i}, X) = \frac{n_{-i,k}}{Z_i (\alpha+n-1)} \int p(x+i|\theta_k) p(\theta_k|X_{-i,k}) d\theta_k$$
$$ X_{-i,k} = (x_1, \ldots, X_{i-1}, X_{i+1}, \ldots, X_k) $$

\end{document}